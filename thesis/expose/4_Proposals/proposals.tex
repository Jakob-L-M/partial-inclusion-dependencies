% In this document I will collect all open questions which I want to investigate during my thesis
To conclude my expose, I will now collect the precise research questions that my thesis will answer. These questions might evolve during the process of writing the actual thesis. There are also points which are not really questions but also tasks I would like to tackle in my thesis.

\begin{itemize}
    \item[1.] Where can we find partial inclusion dependencies in real world data?
    \begin{itemize}
        \item[1.1.] Can we find a threshold that works over a broad range of data sets?
        \item[1.2.] What are the new insights we gain from discovering the partial INDs?
        \item[1.3.] Can we find a group/kind of data sets that commonly includes partial INDs 
    \end{itemize}
    \item[2.] How expensive is meta data?
    \begin{itemize}
        \item[2.1.] I want to provide an in depth analysis on the cost of meta data extraction as well as the time saved by possible invalidations that can be computed using the meta data.
    \end{itemize}
    \item[3.] How much more time does pIND discovery take compared to IND discovery?
    \begin{itemize}
        \item[3.1.] Since pIND discovery is a more complex problem, compared to IND discovery, I assume the algorithms will be slower. I want to investigate how much the execution time is dependent on the threshold $\rho$.
        \item[3.2.] Further I want to find clear statements about which strategies that have been proposed in past literature could be applied in a partial setting and which strategies will not work.
    \end{itemize}
    \item[4.] I want to investigate how a more complex graph can help the discovery process.
    \begin{itemize}
        \item[4.1.] Using the graph I would like to try to optimize the testing order to perform as many (in) validation without needing to test against the actual data.
        \item[4.2.] I want to experiment which complexity (e.g. which values to store in the graph) is computationaly the most efficient. 
    \end{itemize}
    \item[5.] Rewrite BINDER to enable partial IND discovery. This is thought as a benchmark as well as a verification algorithm.
    \item[6.] Extent Metanome, such that partial INDs can be returned properly.
    \item[7.] Find a clear notation for all definitions and provide a well structured basis for partial inclusion dependencies.
    \item[8.] We will analyse how a changing threshold affects the run time, candidate space and pruning possibilities.
\end{itemize}