\documentclass
[
    twoside,                 % The thesis is formatted like a book. That is, odd and even pages are handled differently.
    openright,               % Starts a new chapter on an odd page number (right side).
    cleardoublepage = empty, % Clear pages inserted in order to have new chapters appear on odd pages are formatted with an empty style.
    fontsize = 12 pt,        % The size of the font.
    american,                % Support for American English.
    captions = tableheading, % Places the correct amount of space when the caption of a table is above the table.
    numbers = noenddot,      % Does not use a period at the end of numbered titles, such as sections or figures.
    footheight = 35 pt,      % Defines the height of the foot. Due to the line, it needs extra height.
%    draft,                   % Only displays boxes of figures. This option is useful if compilation takes a long time.
]
{scrbook}


% This file contains all sorts of commands that are used in order to specify certain options for the document.

\newif\ifprintVersion   % Defines a binary variable that signals whether the document is prepared for physical or digital print.
\newif\ifprofessionalPrint % Defines a binary variable that signals whether the print will be done by a professional printing service that requests extra margin for page cutting and is not bound to paper formats like A4.
\newif\iffancyTheorems  % Defines a binary variable that signals whether theorems are formatted in the classical style or in a new format that better suits the overall flavor of this thesis.
\newif\ifboldNumberSets % Defines a binary variable that signals whether the variables for number sets (like N or R) should be in bold. If not, they are in blackboard bold instead.
\newif\ifbachelorThesis % Defines a binary variable that signals whether this thesis is a bachelor thesis (true) or a master thesis (false).

% Set all variables to their default values.
\printVersionfalse
\professionalPrintfalse
\fancyTheoremstrue
\boldNumberSetstrue
\bachelorThesistrue

%%%%%%%%%%%%%%%%%%%%%%%%
% The following commands define certain strings that provide important information for the document.

% The title of the thesis.
\newcommand*{\printTitle}{}
\newcommand*{\printGermanTitle}{}
\newcommand*{\myTitle}[2]{\renewcommand*{\printTitle}{#1}\renewcommand*{\printGermanTitle}{#2}}
\newcommand*{\printTitleBold}{\textbf{\printTitle}}

% The author’s name.
\newcommand*{\printAuthor}{}
\newcommand*{\myName}[1]{\renewcommand*{\printAuthor}{#1}}

% The name of the author’s program.
\newcommand*{\printProgram}{}
\newcommand*{\myProgram}[1]{\renewcommand*{\printProgram}{#1}}

% The date when the thesis was handed in.
\newcommand*{\printDateReceived}{}
\newcommand*{\dateOfHandingIn}[1]{\renewcommand*{\printDateReceived}{#1}}

% A short description of the topic of the thesis. This string will be used for the PDF metadata.
\newcommand*{\printSubject}{}
\newcommand*{\mySubject}[1]{\renewcommand*{\printSubject}{#1}}

% A short description of the topic of the thesis. This string will be used for the PDF metadata.
\newcommand*{\printKeywords}{}
\newcommand*{\myKeywords}[1]{\renewcommand*{\printKeywords}{#1}}

% The name of the author’s supervisor.
\newcommand*{\printNameOfSupervisor}{}
\newcommand*{\nameOfMySupervisor}[1]{\renewcommand*{\printNameOfSupervisor}{#1}}

% The list with the name of the additional examiners.
\newcommand*{\printAdditionalExaminers}{}
\newcommand*{\additionalExaminers}[1]{\renewcommand*{\printAdditionalExaminers}{#1}}

% Defines the extra length added to each side for the print version.
\newlength{\extraborderlength}
\newcommand*{\extraBorder}[1]{\setlength{\extraborderlength}{#1}}

% Defines the length of the binding correction. (The class ›scrbook‹ has a binding correction but it does not work due to all the other packages that are loaded.)
\newlength{\mybindingcorrection}
\newcommand*{\bindingCorrection}[1]{\setlength{\mybindingcorrection}{#1}} % Contains commands that are used for certain information that is printed.


%%%%%%%%%%%%%%%%%%%%%%%%%%%%%%%%%%%%%%
%% Please adjust your options here. %%
%%%%%%%%%%%%%%%%%%%%%%%%%%%%%%%%%%%%%%

    % This section contains commands with important data for your thesis. Please adjust them in order for the document to be printed correctly.

    % Defines the length of the amount that a printed page will be cut from EACH side (including the inner side). This option only takes effect with \printVersiontrue and \professionalPrinttrue.
    \extraBorder{3 mm}

    % Shifts the inner margin outward by the amount specified. When the book is bound, part of the page will not be seen anymore. This option compensates for this loss. It only takes effect with \printVersiontrue.
    \bindingCorrection{6 mm}

    % Use the following command if this is a master thesis.
%    \bachelorThesisfalse

%    \printVersiontrue      % Use this value if you want to prepare your thesis for physical printing. In this case, links will not be colored. Without \professionalPrinttrue, the content will be moved outward by the binding correction, increasing the inner margin and decreasing the outer margin.
%    \professionalPrinttrue % Use this value if you want to have extra border for cutting and are not bound to paper formats (like A4). This option will increase the page size by the extra border on every side plus the binding correction once for the width. It only takes effect in combination with \printVersiontrue.
%    \fancyTheoremsfalse  % Use this value if you want to use the classical theorem style, where the text is italic. Further, with this style, the QED symbol is colorless.
%    \boldNumberSetsfalse % Use this value if you want variables for number sets (like N or R) to appear in blackboard bold rather than bold.

    % The title of the thesis. The first argument is for the English name, the second is for the German name.
    \myTitle{Partial Inclusion Dependency Discovery}{Partial Inclusion Dependency Discovery}

    % The author’s name.
    \myName{Jakob Leander Müller}

    % The author’s program.
    \myProgram{Data Science}

    % The date when the thesis will be handed in.
    \dateOfHandingIn{XX. Februar 2024}

     % The name and affiliation of the author’s supervisor.
    \nameOfMySupervisor{Prof.\,Dr. Thorsten Papenbrock}

    % A list with the names of the additional examiners.
    \additionalExaminers{A Postdoc\newline A PhD student}

    % A short summary of the thesis. These information will be used for the PDF meta data.
    \mySubject{A cool bachelor/master thesis.}

    % Some keywords of the thesis. These information will be used for the PDF meta data. Please use | as a separator and try to avoid commas.
    \myKeywords{master thesis | data profiling | inclusion dependencies}

%%%%%%%%%%%%%%%%%%%%%%%%%%%%%%%%%%%%%%
%% End of options to adjust. %%%%%%%%%
%%%%%%%%%%%%%%%%%%%%%%%%%%%%%%%%%%%%%%


% This file includes all of the code that is used to format the thesis.
% Some packages are included if they are needed. This is done in the respective part and not at the beginning of this file.
%
% This file contains the following parts:
%   • Language an Character Set
%   • Penalties
%   • Indentation
%   • Footnotes
%   • Colors
%   • Size and Position of the Text Body
%   • Position of the Head and the Foot
%   • Margin Position and Width
%   • Header and Footer Format
%   • Caption Format
%   • Part Format
%   • Chapter Format
%   • Table of Contents


%%%%%%%%%%%%%%%%%%%%%%%%%%%%%%%%
%% Language and Character Set %%
%%%%%%%%%%%%%%%%%%%%%%%%%%%%%%%%

\usepackage[utf8]{inputenc} % Allows to input UTF8 characters.
\usepackage[T1]{fontenc}    % Allows to print special characters correctly.
\usepackage
[
    ngerman,         % German is used for the German abstract.
    main = american, % This is the main language of the thesis.
]
{babel}                     % Is responsible for sensible hyphenations.

% The following two commands make it such that the LaTeX compiler of Overleaf produces a PDF from with ligatures and mathematical symbols can be copied correctly.
% Also refer to: https://tex.stackexchange.com/questions/64188/what-are-good-ways-to-make-pdflatex-output-copy-and-pasteable
\input glyphtounicode
\pdfgentounicode=1


%%%%%%%%%%%%%%%
%% Penalties %%
%%%%%%%%%%%%%%%

\widowpenalties 2 10000 0


%%%%%%%%%%%%%%%%%
%% Indentation %%
%%%%%%%%%%%%%%%%%

\usepackage{calc} % Makes it easer to do math with TeX measurements.

\newlength{\myparindent}
\newlength{\myparskip}
\setlength{\myparindent}{1 em}
\setlength{\myparskip}{0 em}

\setlength{\parindent}{\myparindent}
\setlength{\parskip}{\myparskip}
\setlength{\parskip}{0 pt plus 1 pt minus 0 pt}


%%%%%%%%%%%%%%%
%% Footnotes %%
%%%%%%%%%%%%%%%

% Remove the footnote rule.
\setfootnoterule{0 cm}

% The footnote number is made bold and not in superscript.
\deffootnote[1.2 em]{1.2 em}{0 em}{\makebox[1.4 em][l]{\textbf{\thefootnotemark}}}

% The footnote number will not be reset after every chapter.
\makeatletter%
    \@removefromreset{footnote}{chapter}%
\makeatother


%%%%%%%%%%%%
%% Colors %%
%%%%%%%%%%%%

\usepackage[dvipsnames]{xcolor} % Allows it to define colors. The option says that common names can be used.

% Dark blue.
\definecolor{stroke1}{HTML}{2574A9} % This color is used as the standard color to highlight things.


% Coloring various different labels.
\colorlet{captionlabel}{black}
\colorlet{footerpagenr}{black}
\colorlet{footerchapter}{stroke1}
\colorlet{footerchaptername}{black}
\colorlet{footersection}{stroke1}
\colorlet{footersectionname}{black}
\colorlet{chapternumber}{stroke1}


%%%%%%%%%%%%%%%%%%%%%%%%%%%%%%%%%%%%%%%%
%% Size and Position of the Text Body %%
%%%%%%%%%%%%%%%%%%%%%%%%%%%%%%%%%%%%%%%%

% The new paper dimensions that are exclusively used.
\newlength{\mypaperwidth}
\setlength{\mypaperwidth}{210 mm}

\newlength{\mypaperheight}
\setlength{\mypaperheight}{297 mm}

% The text area uses aesthetically pleasing measurements in the same ratio as the page.
% These dimensions are always used, as the text area should be the same in the printed and digital version of the thesis.
\newlength{\mybodywidth}
\setlength{\mybodywidth}{140 mm}

\newlength{\mybodyheight}
\setlength{\mybodyheight}{198 mm}

\newlength{\myoutermargin}
\ifprintVersion
    \ifprofessionalPrint
        \setlength{\myoutermargin}{(\mypaperwidth - \mybodywidth) / \real{1.5} + \extraborderlength}
    \else
        \setlength{\myoutermargin}{(\mypaperwidth - \mybodywidth) / \real{1.5} - \mybindingcorrection}
    \fi
\else
    \setlength{\myoutermargin}{(\mypaperwidth - \mybodywidth) / \real{1.5}}
\fi

\newlength{\mytopmargin}
\setlength{\mytopmargin}{(\mypaperheight - \mybodyheight) / 3}
\ifprintVersion
    \ifprofessionalPrint
        \setlength{\mytopmargin}{(\mypaperheight - \mybodyheight) / 3 + \extraborderlength}
    \fi
\fi

\newlength{\myinnermargin}
\setlength{\myinnermargin}{\mypaperwidth - \mybodywidth - \myoutermargin}
\ifprintVersion
    \ifprofessionalPrint
        \setlength{\myinnermargin}{\mypaperwidth + \mybindingcorrection + 2\extraborderlength - \mybodywidth - \myoutermargin}
    \fi
\fi

\newlength{\mybottommargin}
\setlength{\mybottommargin}{\mypaperheight - \mybodyheight - \mytopmargin}
\ifprintVersion
    \ifprofessionalPrint
        \setlength{\mybottommargin}{\mypaperheight + 2\extraborderlength - \mybodyheight - \mytopmargin}
    \fi
\fi


%%%%%%%%%%%%%%%%%%%%%%%%%%%%%%%%%%%
%% Position of the Head And Foot %%
%%%%%%%%%%%%%%%%%%%%%%%%%%%%%%%%%%%

\newcommand{\goldenratio}{1.618}

\newlength{\myheadsep} % Distance from the header to the body.
\setlength{\myheadsep}{\mytopmargin / \real{\goldenratio} / \real{\goldenratio} - 1 ex}
\ifprintVersion
    \ifprofessionalPrint
        \setlength{\myheadsep}{(\mytopmargin - \extraborderlength) / \real{\goldenratio} / \real{\goldenratio} - 1 ex}
    \fi
\fi

\newlength{\myfootskip} % Distance from the body to the footer.
\setlength{\myfootskip}{\mybottommargin / \real{\goldenratio} - 1 ex}
\ifprintVersion
    \ifprofessionalPrint
        \setlength{\myfootskip}{(\mybottommargin - \extraborderlength) / \real{\goldenratio} - 1 ex}
    \fi
\fi


%%%%%%%%%%%%%%%%%%%%%%%%%%%%%%%
%% Margin Position And Width %%
%%%%%%%%%%%%%%%%%%%%%%%%%%%%%%%

\newlength{\mymargininnersep} % Distance between the margin and the body.
\setlength{\mymargininnersep}{7 mm}

\newlength{\mymarginoutersep} % Distance between the margin and the paper border.
\setlength{\mymarginoutersep}{12 mm}
\ifprintVersion
    \ifprofessionalPrint
        \setlength{\mymarginoutersep}{12 mm + \extraborderlength}
    \fi
\fi

\newlength{\mymarginwidth} % Width of the margin.
\setlength{\mymarginwidth}{\myoutermargin - \mymargininnersep - \mymarginoutersep}

\newlength{\mymarginwidthwithinnersep} % Width of the margin.
\setlength{\mymarginwidthwithinnersep}{\mymarginwidth + \mymargininnersep}

\usepackage
[
    % In the printed version, we add an extra border to each side as well as the binding correction for the width.
    \ifprintVersion
        \ifprofessionalPrint
            paperwidth = \mypaperwidth + 2\extraborderlength + \mybindingcorrection,
            paperheight = \mypaperheight + 2\extraborderlength,
        \else
            paperwidth = \mypaperwidth,
            paperheight = \mypaperheight,
        \fi
    \else
        paperwidth = \mypaperwidth,
        paperheight = \mypaperheight,
    \fi
    textwidth = \mybodywidth,
    textheight = \mybodyheight,
    outer = \myoutermargin,
    top = \mytopmargin,
    headsep = \myheadsep,
    footskip = \myfootskip,
    marginparsep = \mymargininnersep,
    marginparwidth = \mymarginwidth,
%    showframe, % Use this option for debugging purposes in order to the an outline of all of the different parts of the page layout.
]
{geometry} % Used in order to define the dimensions of the page and its layout.


%%%%%%%%%%%%%%%%%%%%%%%%%%%%%%
%% Header and Footer Format %%
%%%%%%%%%%%%%%%%%%%%%%%%%%%%%%

\usepackage
[
%    draft, % Shows a lot of rules denoting the dimensions of the head and foot. Use this option only for debugging.
]
{scrlayer-scrpage} % Allows to adjust the definitions of the head and foot of a page.

% Remove all predefined styles.
\clearpairofpagestyles

%%%%%%%%%%%%%%%%%%%%%%%%%%%%%%
% Dimensions and formats are defined.

% Define the dimensions of the head and the foot. Since we want some information to appear in the margin, we extend the head and the foot by the respective lengths.
\KOMAoptions
{%
    headwidth = \textwidth + \mymarginwidthwithinnersep,%
    footwidth = \myoutermargin : \textwidth,%
}

% Defines the formats for the chapter and section titles in the marks of the head.
\renewcommand*{\chaptermarkformat}{\normalfont\sffamily\small\color{footerchaptername}}
\renewcommand*{\sectionmarkformat}{\normalfont\sffamily\small\color{footersectionname}}

% Displays the chapter names in the head of both odd and even pages.
\automark[chapter]{chapter}
% Replaces the chapter name to the head of right pages with the section name if a section is present.
\automark*[section]{}

%%%%%%%%%%%%%%%%%%%%%%%%%%%%%%
% The head is defined.

% Head for even pages.
% Puts ›Chapter‹ followed by the current chapter number.
\lehead%
{%
    \begin{minipage}[b]{\mymarginwidth}%
        \small\raggedleft\normalfont\textsf{\textbf{\color{footerchapter}\chaptername\ \thechapter}}
    \end{minipage}
}
% Put the title of the current chapter/section into the center of the head but push it to the border.
\cehead{\hspace*{\mymarginwidthwithinnersep}\parbox{\textwidth}{\raggedright\leftmark}}

% Head for odd pages.
\rohead%
{%
    % Check whether a section has already started or not.
    \Ifstr{\rightmark}{\leftmark}%
    {%
        \begin{minipage}[b]{\mymarginwidth}%
            \small\raggedright\normalfont\textsf{\textbf{\color{footersection}Chapter\ \thechapter}}%
        \end{minipage}%
    }%
    {%
        \begin{minipage}[b]{\mymarginwidth}%
            \small\raggedright\normalfont\textsf{\textbf{\color{footersection}Section\ \thesection}}%
        \end{minipage}%
    }%
}
\cohead{\hspace*{-\mymarginwidthwithinnersep}\parbox{\textwidth}{\raggedleft\rightmark}}

%%%%%%%%%%%%%%%%%%%%%%%%%%%%%%
% The foot is defined.

% Displays the page number in bold in the margin, aligned toward the center. Further, a blue line is drawn above number.
% The starred variant is used, since we want the format of the foot to also apply to the pagestyle ›plain‹.
\lefoot*%
{%
    \vspace*{1 ex}%
    {\color{stroke1}\rule{\myoutermargin - \mymargininnersep}{0.5 mm}}\\
    \begin{minipage}[b]{\myoutermargin - \mymargininnersep}%
        \raggedleft\normalfont\color{footerpagenr}\textbf{\thepage}%
    \end{minipage}%
}
\rofoot*%
{%
    {\color{stroke1}\rule{\myoutermargin - \mymargininnersep}{0.5 mm}}\\
    \begin{minipage}[b]{\myoutermargin - \mymargininnersep}%
        \raggedright\normalfont\color{footerpagenr}\textbf{\thepage}%
    \end{minipage}%
}


%%%%%%%%%%%%%%%%%%%%
%% Caption Format %%
%%%%%%%%%%%%%%%%%%%%

\usepackage{caption}
\captionsetup
{
    font = small,
    labelfont = {bf, sf, color = captionlabel},
    format = plain,
    singlelinecheck = off,
}


%%%%%%%%%%%%%%%%%
%% Part Format %%
%%%%%%%%%%%%%%%%%
\usepackage{tikz} % Used in order to draw the stylistic elements.

\newlength{\mytmpa}
\setlength{\mytmpa}{1 mm}
\newlength{\mytmpb}
\newlength{\mytmpc}

%%%%%%%%%%%%%%%%%
% The following code draws the outline for a ›part‹ of the thesis.
% This command is used before the name of the part is displayed. It is void, as the part is added via \partlineswithprefixformat.
\renewcommand*{\partformat}{}
% This command calls \partformat (#2) and displays the name of the part (#3).
\renewcommand*{\partlineswithprefixformat}[3]%
{%
    #2
    \thispagestyle{empty}
    \setlength{\mytmpa}{0.618\mypaperwidth}%
    \setlength{\mytmpb}{0.382\mypaperheight}%
    \ifprintVersion
        \ifprofessionalPrint
            \setlength{\mytmpa}{0.618\mypaperwidth + \mybindingcorrection + \extraborderlength}%
            \setlength{\mytmpb}{0.382\mypaperheight + \extraborderlength}%
        \fi
    \fi
    \begin{tikzpicture}[overlay, remember picture]%
        \node [inner sep = 0, outer sep = 0, anchor = north] at (current page.north west)%
        {%
            \begin{tikzpicture}[overlay, remember picture]%
            \draw[color = stroke1, line width = 0.7 mm] (\mytmpa, 0) -- (\mytmpa, -\mytmpb);%
            \end{tikzpicture}%
        };%
        \node (align) [align = right, below = \mytmpb - 2 ex, inner sep = 0, outer sep = 0, anchor = north west] at (current page.north west)%
        {%
            \hspace{\mytmpa}\hspace{0.5 em}\partname\ \thepart\\[1 ex]
            \color{stroke1}#3%
        };%
    \end{tikzpicture}%
}
% This command defines various parameters for the ›part‹ format.
\RedeclareSectionCommand%
[%
    font = \normalfont\Huge\sffamily,
    prefixfont = \normalfont\Huge\sffamily,
]
{part}


%%%%%%%%%%%%%%%%%%%%%
%%% Chapter Format %%
%%%%%%%%%%%%%%%%%%%%%

\usepackage{etoolbox}

\newbool{chapterHasANumber}
\newbool{chapterHasAStar}
\renewcommand*{\chapterlinesformat}[3]%
{%
    % Check whether \chapter of \addchap has been used.
    \Ifnumbered{#1}{\setbool{chapterHasANumber}{true}}{\setbool{chapterHasANumber}{false}}%
    % Check whether \chapter* or \chapter has been used.
    \Ifstr{#2}{}{\setbool{chapterHasAStar}{true}}{\setbool{chapterHasAStar}{false}}%
    % Check whether a normal \chapter or something else is used.
    \ifboolexpr{bool{chapterHasANumber} and not bool{chapterHasAStar}}%
    {%
        \begin{tikzpicture}[overlay, remember picture]%
            \node [right = \myinnermargin, below = \mytopmargin, inner sep = 0, outer sep = 0, anchor = north west] (numbernode) at (current page.north west)%
            {%
                \hspace{\myinnermargin}%
                \sffamily\fontsize{60}{60}\selectfont%
                \color{chapternumber}%
                \thechapter%
            };%
            \node [inner sep = 0, outer sep = 0, anchor = north west] at (numbernode.south west)%
            {%
                \begin{tikzpicture}[overlay, remember picture]%
                    \draw[color = stroke1, line width = 0.7 mm] (\myinnermargin, -1 ex) -- (\paperwidth, -1 ex);%
                \end{tikzpicture}%
            };%
            \node (align) [text width = \textwidth - 2 cm, align = right, right = \myinnermargin + \mybodywidth, inner sep = 0, outer sep = 0, anchor = east] at (numbernode.west)%
            {%
                #3%
            };%
        \end{tikzpicture}%
    }%
    {%
        \begin{tikzpicture}[overlay, remember picture]%
            \node [right = \myinnermargin, below = \mytopmargin, inner sep = 0, outer sep = 0, anchor = north west] (numbernode) at (current page.north west)%
            {%
                \hspace{\myinnermargin}%
                \sffamily\fontsize{60}{60}\selectfont%
                \color{white}%
                \thechapter%
            };%
            \node [inner sep = 0, outer sep = 0, anchor = north west] at (numbernode.south west)%
            {%
                \begin{tikzpicture}[overlay, remember picture]%
                    \draw[color = stroke1, line width = 0.7 mm] (\myinnermargin, -1 ex) -- (\paperwidth, -1 ex);%
                \end{tikzpicture}%
            };%
            \node (align) [align = left, right = \myinnermargin, inner sep = 0, outer sep = 0, anchor = south west] at (numbernode.south west)%
            {%
                #3%
            };%
        \end{tikzpicture}%
    }%
}
\RedeclareSectionCommand%
[%
    font = \color{stroke1}\normalfont\huge\sffamily,
    afterskip = 20 pt,
]
{chapter}


%%%%%%%%%%%%%%%%%%%%%%%%
%%% Table of Contents %%
%%%%%%%%%%%%%%%%%%%%%%%%

% Format the table of contents to have a ›plain‹ page style.
\BeforeStartingTOC[toc]{\pagestyle{plain}}
\AfterStartingTOC{\thispagestyle{plain}}                        % Contains commands that define the general format and layout of the thesis.
% This file contains all of the code that formats the bibliography. Since be package ›biblatex‹ is used, the bibliography needs to be compiled with ›biber‹.
%
% This file contains the following parts:
%   • Resources
%   • Redefined Keywords
%   • Coloring
%   • Format of the Entries
%   • Format of the Own Publications


\usepackage
[
    sortcites,              % Sort multiple references when citing them together.
    style = alphabetic,     % The style of a citation mark.
    defernumbers,           % Makes sure that references always have unique numbers. This is important if you use multiple bibliographies.
    safeinputenc,           % Allows to use UTF8 characters in the bibliography and tries to translate them into TeX automatically.
    backref = true,         % Creates back references in the bibliography.
    backrefstyle = three,   % Compresses three or more consecutive pages in the back references into a range.
    hyperref = true,        % Makes links generated by biblatex clickable. If hyperref is not used, a warning is issued.
    maxbibnames = 99,       % The maximum number of names displayed in the bibliography.
    maxcitenames = 2,       % The maximum number of names displayed when using commands like ›textcite‹. The default is 3. After that, ›et al.‹ is used.
%    useprefix,              % Prints name prefixes, such as ›von‹. The default is false. This means that prefixes are not considered to be part of the last name.
]
{biblatex} % Used in order to format the bibliography.

% The following command changes the space between the list of authors and the citation mark into a non-breaking space.
\renewcommand\namelabeldelim{\addnbspace}


%%%%%%%%%%%%%%%
%% Resources %%
%%%%%%%%%%%%%%%

\addbibresource{references/strings.bib}                     % Contains many strings for common conference names etc. These strings can then be used in the references.
\addbibresource{references/references.bib}                  % The file that contains the references that are used for the thesis.


%%%%%%%%%%%%%%%%%%%%%%%%
%% Redefined Keywords %%
%%%%%%%%%%%%%%%%%%%%%%%%

\renewbibmacro{in:}%
{%
    \ifentrytype{article}{}{\printtext{\bibstring{in}\intitlepunct}}%
}
% \renewcommand*{\volumenumberdelim}{\addcolon}

\renewbibmacro*{volume+number+eid}%
{%
    \printfield{volume}%
    \iffieldundef{number}{}{\addcolon}%
    %  \setunit*{\addnbthinspace}%
    \printfield{number}%
    \setunit*{\addcomma\space}%
    \printfield{eid}%
}

\DefineBibliographyStrings{english}%
{%
    backrefpage  = {\lowercase{s}ee page}, % For a single page number.
    backrefpages = {\lowercase{s}ee pages} % For multiple page numbers.
}


%%%%%%%%%%%%%%
%% Coloring %%
%%%%%%%%%%%%%%

\DeclareFieldFormat[article]{title}{\textbf{\color{stroke1}#1}}
\DeclareFieldFormat[inproceedings]{title}{\textbf{\color{stroke1}#1}}
\DeclareFieldFormat[thesis]{title}{\textbf{\color{stroke1}#1}}
\DeclareFieldFormat[book]{title}{\textbf{\color{stroke1}#1}}
\DeclareFieldFormat[unpublished]{title}{\textbf{\color{stroke1}#1}}
\DeclareFieldFormat[report]{title}{\textbf{\color{stroke1}#1}}
\DeclareFieldFormat[inbook]{chapter}{\textbf{\color{stroke1}#1}}
\DeclareFieldFormat[inbook]{title}{#1}
\DeclareFieldFormat{pages}{#1}


%%%%%%%%%%%%%%%%%%%%%%%%%%%
%% Format of the Entries %%
%%%%%%%%%%%%%%%%%%%%%%%%%%%

% The following toggle defines how the citation mark formats the author names. If this toggle is true, more information is used.
\newtoggle{authorend}
\togglefalse{authorend}

% Article
\DeclareBibliographyDriver{article}%
{%
  \usebibmacro{bibindex}%
  \usebibmacro{begentry}%
  \iftoggle{authorend}{}{\usebibmacro{author/translator+others}}%
  \setunit{\labelnamepunct}\newblock
  \usebibmacro{title}%
  \newunit
  \printlist{language}%
  \newunit\newblock
  \usebibmacro{byauthor}%
  \newunit\newblock
  \usebibmacro{bytranslator+others}%
  \newunit\newblock
  \printfield{version}%
  \newunit\newblock
  \usebibmacro{in:}%
  \usebibmacro{journal+issuetitle}%
  \newunit
  \usebibmacro{byeditor+others}%
  \newunit
  \usebibmacro{note+pages}%
  \newunit\newblock
  \iftoggle{bbx:isbn}
  {\printfield{issn}}
  {}%
  \newunit\newblock
  \usebibmacro{doi+eprint+url}%
  \newunit\newblock
  \usebibmacro{addendum+pubstate}%
  \setunit{\bibpagerefpunct}\newblock
  \usebibmacro{pageref}%
  \newunit\newblock
  \iftoggle{bbx:related}
  {\usebibmacro{related:init}%
    \usebibmacro{related}}
  {}%
  \usebibmacro{finentry}%
  \iftoggle{authorend}{\usebibmacro{author/translator+others}}{}%
}

% Book Chapter
\DeclareBibliographyDriver{inbook}%
{%
  \usebibmacro{bibindex}%
  \usebibmacro{begentry}%
  \iftoggle{authorend}{}{\usebibmacro{author/translator+others}}%
  \setunit{\labelnamepunct}\newblock
  % \usebibmacro{title}%
  \usebibmacro{chapter+pages}%
  % \printfield{chapter}%
  \newunit
  \printlist{language}%
  \newunit\newblock
  \usebibmacro{byauthor}%
  \newunit\newblock
  \usebibmacro{in:}%
  \usebibmacro{bybookauthor}%
  \newunit\newblock
  \usebibmacro{maintitle+booktitle}%
  \newunit\newblock
  \usebibmacro{byeditor+others}%
  \newunit\newblock
  \printfield{edition}%
  \newunit
  \iffieldundef{maintitle}
  {\printfield{volume}%
    \printfield{part}}
  {}%
  \newunit
  \printfield{volumes}%
  \newunit\newblock
  \usebibmacro{series+number}%
  \newunit\newblock
  \printfield{note}%
  \newunit\newblock
  \usebibmacro{publisher+location+date}%
  \newunit\newblock
  % \usebibmacro{chapter+pages}%
  \newunit\newblock
  \iftoggle{bbx:isbn}
  {\printfield{isbn}}
  {}%
  \newunit\newblock
  \usebibmacro{doi+eprint+url}%
  \newunit\newblock
  \usebibmacro{addendum+pubstate}%
  \setunit{\bibpagerefpunct}\newblock
  \usebibmacro{pageref}%
  \newunit\newblock
  \iftoggle{bbx:related}
  {\usebibmacro{related:init}%
    \usebibmacro{related}}
  {}%
  \usebibmacro{finentry}%
  \iftoggle{authorend}{\usebibmacro{author/translator+others}}{}%
}

% Proceedings Article
\DeclareBibliographyDriver{inproceedings}%
{%
  \usebibmacro{bibindex}%
  \usebibmacro{begentry}%
  \iftoggle{authorend}{}{\usebibmacro{author/translator+others}}%
  \setunit{\labelnamepunct}\newblock
  \usebibmacro{title}%
  \newunit
  \printlist{language}%
  \newunit\newblock
  \usebibmacro{byauthor}%
  \newunit\newblock
  \usebibmacro{in:}%
  \usebibmacro{maintitle+booktitle}%
  \newunit\newblock
  \usebibmacro{event+venue+date}%
  \newunit\newblock
  \usebibmacro{byeditor+others}%
  \newunit\newblock
  \iffieldundef{maintitle}
  {\printfield{volume}%
    \printfield{part}}
  {}%
  \newunit
  \printfield{volumes}%
  \newunit\newblock
  \usebibmacro{series+number}%
  \newunit\newblock
  \printfield{note}%
  \newunit\newblock
  \printlist{organization}%
  \newunit
  \usebibmacro{publisher+location+date}%
  \newunit\newblock
  \usebibmacro{chapter+pages}%
  \newunit\newblock
  \iftoggle{bbx:isbn}
  {\printfield{isbn}}
  {}%
  \newunit\newblock
  \usebibmacro{doi+eprint+url}%
  \newunit\newblock
  \usebibmacro{addendum+pubstate}%
  \setunit{\bibpagerefpunct}\newblock
  \usebibmacro{pageref}%
  \newunit\newblock
  \iftoggle{bbx:related}
  {\usebibmacro{related:init}%
    \usebibmacro{related}}
  {}%
  \usebibmacro{finentry}%
  \iftoggle{authorend}{\usebibmacro{author/translator+others}}{}%
}

% Thesis
\DeclareBibliographyDriver{thesis}%
{%
  \usebibmacro{bibindex}%
  \usebibmacro{begentry}%
  \iftoggle{authorend}{}{\usebibmacro{author}}%
  \setunit{\labelnamepunct}\newblock
  \usebibmacro{title}%
  \newunit
  \printlist{language}%
  \newunit\newblock
  \usebibmacro{byauthor}%
  \newunit\newblock
  \printfield{note}%
  \newunit\newblock
  \printfield{type}%
  \newunit
  \usebibmacro{institution+location+date}%
  \newunit\newblock
  \usebibmacro{chapter+pages}%
  \newunit
  \printfield{pagetotal}%
  \newunit\newblock
  \iftoggle{bbx:isbn}
  {\printfield{isbn}}
  {}%
  \newunit\newblock
  \usebibmacro{doi+eprint+url}%
  \newunit\newblock
  \usebibmacro{addendum+pubstate}%
  \setunit{\bibpagerefpunct}\newblock
  \usebibmacro{pageref}%
  \newunit\newblock
  \iftoggle{bbx:related}
  {\usebibmacro{related:init}%
    \usebibmacro{related}}
  {}%
  \usebibmacro{finentry}%
  \iftoggle{authorend}{\usebibmacro{author}}{}%
}


%%%%%%%%%%%%%%%%%%%%%%%%%%%%%%%%%%%%
%% Format of the Own Publications %%
%%%%%%%%%%%%%%%%%%%%%%%%%%%%%%%%%%%%

% The own publications are formatted using a numeric list, whereas the bibliography of the thesis uses an alphanumeric style.

% Copied from numeric.cbx in order to imitate numerical citations.
\providebool{bbx:subentry}
\newbibmacro*{citenum}%
{% Note: the original macro was called ›cite‹. I did not redefine ›cite‹ but instead defined a new macro ›citenum‹ because the author-year citations use the ›cite‹ macro too. Using ›\renewbibmacro*{cite}‹ would have caused all the author-year citations to become numeric too.
  \printtext[bibhyperref]{% If you ever want to use hyperref.
    \printfield{prefixnumber}%
    \printfield{labelnumber}%
    \ifbool{bbx:subentry}
    {\printfield{entrysetcount}}
    {}}%
}

% Copied from numeric.cbx to define a new numeric citation command for @online entries.
\DeclareCiteCommand{\conline}[\mkbibbrackets]
{\usebibmacro{prenote}}
{\usebibmacro{citeindex}%
  \usebibmacro{citenum}}% Note: this was originally "cite" but I changed it to "citenum" to avoid clashes with the author-year style.
{\multicitedelim}
{\usebibmacro{postnote}}       % Contains commands for the layout of the bibliography.
% This file contains most of the packages used for this document. If you want to add a package, do it here.
% Some packages are already included in other files in the ›core‹ folder if they were already necessary. Thus, make sure to go through these files too if you want to know whether a certain package is already included.
%
% This file contains the following parts:
%   • Typography
%   • Math
%   • Fonts
%   • Graphics
%   • Tables
%   • Enumerations
%   • Algorithms
%   • Spaces and Special Characters
%   • Miscellaneous
%   • Additional Packages
%   • Hyperlinks

%%%%%%%%%%%%%%%%
%% Typography %%
%%%%%%%%%%%%%%%%

\usepackage
[
    babel = true, % Enables language-specific tuning.
]
{microtype}           % Uses the text space more efficiently.
\usepackage{csquotes} % Uses the correct quotes according to the current language.


%%%%%%%%%%
%% Math %%
%%%%%%%%%%

% The following packages are the standard packages used in order to typeset math. They contain a lot of useful commands.
\usepackage{amsmath}
\usepackage{amssymb}
\usepackage{amsthm}
\usepackage{thmtools}
\usepackage{mathtools}
\usepackage{thm-restate}
\usepackage{dsfont}        % Yields far better blackboard-bold letters than \mathbb. Use \mathds in order to write such letters.
\usepackage{braceMnSymbol} % Adjusts overbraces and underbraces such that longer versions are put together seamlessly.


%%%%%%%%%%%
%% Fonts %%
%%%%%%%%%%%

\usepackage
[
    ttscale = 0.85, % Scales the typewriter font.
]
{libertine} % The main font used in this thesis.
\usepackage
[
    libertine,    % Changes the math font to libertine (the main font).
    slantedGreek, % Makes all greek letters italic by default. If you want to use an upright greek letter, use ›\up‹ immediately followed by the letter’s name. For example, \upGamma displays an upright uppercase gamma.
    vvarbb,       % Changes the \mathbb font to another font. However, \mathbb remains ugly and should not be used. Use \mathds instead.
    libaltvw,     % Uses different characters for v und w that look far better than the default ones.
]
{newtxmath} % The main math font of this thesis. It fits well with the main font.
\usepackage{url} % Responsible for URL formatting.
\usepackage{bm}  % Allows to use sensible bold letters in math mode. This package has to go after the font packages. Otherwise it does not work correctly!


%%%%%%%%%%%%%%
%% Graphics %%
%%%%%%%%%%%%%%

\usepackage{graphicx} % The standard package for including graphics into your document.
\usepackage
[
    subrefformat = simple, % Formats the label of the \subref command without parentheses.
    labelformat = simple,  % Formats the mark of a subfigure without parentheses.
]
{subcaption}         % Enables it to have subfigures inside of a single figure.
\usepackage{wrapfig} % Allows to put figures next to text.

% Changing the \columnsep adds some space next to a warpfigure.
\columnsep = \mymargininnersep
% The reference label of a subfigure is redefined to have a non-breaking space and parentheses. (Thus, the subfigures show parentheses although the package options removed parentheses; otherwise, two pairs of brackets would be seen.)
\renewcommand*{\thesubfigure}{~(\alph{subfigure})}


%%%%%%%%%%%%
%% Tables %%
%%%%%%%%%%%%

\usepackage{array}     % Improves the way that tables can be formatted.
\usepackage{booktabs}  % Adds lines (called ›rules‹) that can be used in tables and improves spacing.
\usepackage{longtable} % Allows to make tables that span multiple pages.
\usepackage{pdflscape} % Allows to change a page into landscape. This is handy if a table is very wide.


%%%%%%%%%%%%%%%%%%
%% Enumerations %%
%%%%%%%%%%%%%%%%%%

\usepackage{enumitem} % Adds tons of useful features to enumeration environments.


\usepackage{xspace}   % Adds the functionality that a space after a command will be shown as a space in the output.
\usepackage
[
    shortcuts, % Allows to use short symbols for non-breaking hyphens and dashes instead of lengthy commands.
]
{extdash}             % Adds non-breaking hyphens and dashes.
\usepackage{setspace} % Allows to easily chnage the spacing inside of the document.


%%%%%%%%%%%%%%%%%%%
%% Miscellaneous %%
%%%%%%%%%%%%%%%%%%%

\usepackage{xparse}    % Is used in order to define reasonable commands.
\usepackage{footnote}  % Allows it to extend the environments footnotes can be used in. It is said that this package is in conflict with ›hyperref‹. I did not note any troubles. However, if something is fishy, it is probably best to not use this package.
\usepackage{afterpage} % Adds the \afterpage command, which specifies that the provided argument shall be processed after the current page is finished.
\usepackage
[
    textsize = scriptsize, % Determines the text size of the TODO note.
]
{todonotes}            % Adds TODO notes to the document. These are small text areas inside of the margin of a page.


%%%%%%%%%%%%%%%%%%%%%%%%%
%% Additional Packages %%
%%%%%%%%%%%%%%%%%%%%%%%%%

% Add additional packages you would like to use here.
\usepackage{algpseudocode}



%%%%%%%%%%%%%%%%
%% Hyperlinks %%
%%%%%%%%%%%%%%%%

\usepackage
[
    bookmarks = true,                 % Generates boodmarks for the PDF.
    bookmarksopen = false,            % The bookmarks are closed by default.
    bookmarksnumbered = true,         % The bookmarks use the numbers of the corresponding headline.
    pdfstartpage = 1,                 % The first page seen when opening the PDF.
    pdftitle = {{\printTitle}},       % The PDF’s title in the meta data.
    pdfauthor = {{\printAuthor}},     % The PDF’s author name in the meta data.
    pdfsubject = {{\printSubject}},   % The PDF’s subject in the meta data.
    pdfkeywords = {{\printKeywords}}, % The PDF’s keywords in the meta data.
    breaklinks = true,                % Allows it to break links.
    \ifprintVersion
        hidelinks,                    % In the printed version, links are not highlighted, as they are not clickable.
    \else
    colorlinks = true,            % The text of hyperlinks is colored instead of having a colored box around it.
    allcolors = stroke1,          % Every hyperlink uses the same color. If you want to change specific colors, use the commands below.
    %        linkcolor = stroke1,          % The color of an in-document hyperlink.
    %        citecolor = stroke1,          % The color of a citation.
    %        filecolor = stroke1,          % The color of a file link.
    %        pagecolor = stroke1,          % The color of a reference to a page.
    %        urlcolor = stroke1,           % The color of a weblink.
    \fi
]
{hyperref} % The standard package that is used for creating hyperlinks inside of a document.

\usepackage
[
    %    capitalise, % Capitalizes the words in front of the labels. This can also be done by simply using \Cref instead of \cref. In order to have a greater variety, this option is not used.
    noabbrev,   % The words in front of the labels are not abbreviated.
    nameinlink, % Extends the link of a reference to the word in front of it.
]
{cleveref} % This package must be included after ›hyperref‹. It creates clever references that know what they refer to.     % Contains the packages that this template provides.
% This file contains all sorts of macros that are globally used. Further, certain options made available through packages are set here as well.
%
% This file contains the following parts:
%   • Type of Degree
%   • Miscellaneous
%   • Footnotes
%   • Theorem Environments
%   • Meta Commands
%   • Common Commands


%%%%%%%%%%%%%%%%%%%%
%% Type of Degree %%
%%%%%%%%%%%%%%%%%%%%

% The colloquial term of the degree.
\newcommand*{\colloquialDegreeName}{Master}
\newcommand*{\colloquialDegreeNameLowercase}{master}

% The abbreviation of the degree.
\newcommand*{\degreeAbbreviation}{M.}

% Redefine the two macors above for a bachelor thesis.
\ifbachelorThesis
    \renewcommand*{\colloquialDegreeName}{Bachelor}
    \renewcommand*{\colloquialDegreeNameLowercase}{bachelor}
    \renewcommand*{\degreeAbbreviation}{B.}
\fi


%%%%%%%%%%%%%%%%%%%
%% Miscellaneous %%
%%%%%%%%%%%%%%%%%%%

% Defines the environment used at the beginning of each chapter.
\newenvironment{jointwork}
{\itshape}
{\ignorespacesafterend\bigskip}

% Defines the IfEmptyTF command. This is useful for optional arguments provided as [].
\makeatletter
    \def\IfEmptyTF#1%
    {%
        \if\relax\detokenize{#1}\relax%
            \expandafter\@firstoftwo%
        \else%
            \expandafter\@secondoftwo%
        \fi%
    }
\makeatother

% Creates an environment that automatically uses math mode if necessary and creates a space afterward if wanted. Basically, if the command \example is defined to use this environment, you can use \example without mathe mode in normal text as if it were ordinary text.
\NewDocumentCommand{\mathOrText}{m}
{%
    \ensuremath{#1}\xspace%
}

% Reduces the space around scaling bracekts.
\let\originalleft\left
\let\originalright\right
\renewcommand{\left}{\mathopen{}\mathclose\bgroup\originalleft}
\renewcommand{\right}{\aftergroup\egroup\originalright}

% Lets math text in an environment of bold text also appear bold.
\makeatletter
    \DeclareRobustCommand{\bfseries}%
    {%
        \not@math@alphabet\bfseries\mathbf%
        \fontseries\bfdefault\selectfont%
        \boldmath%
    }
\makeatother

% Adds square and curly brackets to the exceptions for xspace such that no space is used right in front of them.
\xspaceaddexceptions{]\}}

% Formats URLs by using the normal font (not the typewriter font).
\urlstyle{rm}

% Allows large display formulas to span multiple pages.
\allowdisplaybreaks

% Defines an optional argument for labels named ›ineq‹ that signals that cleveref should name the respective reference ›inequality‹ instead of its actual name.
\crefname{ineq}{inequality}{inequalities}
\creflabelformat{ineq}{#2{\upshape(#1)}#3} 

% Defines an optional argument for labels named ›term‹ that signals that cleveref should name the respective reference ›term‹ instead of its actual name.
\crefname{term}{term}{terms}
\creflabelformat{term}{#2{\upshape(#1)}#3}


%%%%%%%%%%%%%%%
%% Footnotes %%
%%%%%%%%%%%%%%%

% In the following, the command ›footnote‹ is redefined such that the footnote mark can be more easily adjusted.
\let\oldfootnote\footnote

% The following are variables used by the command.
\newlength{\spaceBeforeFootnote} % Denotes the space before the footnote mark in em.
\newlength{\spaceAfterFootnote}  % Denotes the space after the footnote mark in em.

% The new footnote command. The first three arguments are optional, the fourth mandatory. Its arguments have the following meaning:
%   1. The amount of space before the footnote mark in em. The default is 0.
%   2. The amount of space after the footnote mark in em. The default is 0.
%   3. The number of the footnote mark.
%   4. The text of the footnote.
\RenewDocumentCommand{\footnote}{o o o m}%
{%
    \IfNoValueTF{#1}%
    {%
        \oldfootnote{#4}%
    }%
    {%
        \setlength{\spaceBeforeFootnote}{\IfEmptyTF{#1}{0}{#1} em}%
        \IfNoValueTF{#2}%
        {%
            \hspace*{\spaceBeforeFootnote}\oldfootnote{#4}%
        }%
        {%
            \setlength{\spaceAfterFootnote}{\IfEmptyTF{#2}{0}{#2} em}%
            \hspace*{\spaceBeforeFootnote}\IfNoValueTF{#3}{\oldfootnote{#4}}{\oldfootnote[#3]{#4}}\hspace*{\spaceAfterFootnote}%
        }%
    }%
}

% The following commands enable it such that footnotes can be used in various other environments other than simple text.
\makesavenoteenv{figure}
\makesavenoteenv{table}
\makesavenoteenv{tabular}


%%%%%%%%%%%%%%%%%%%%%%%%%%
%% Theorem Environments %%
%%%%%%%%%%%%%%%%%%%%%%%%%%

\iffancyTheorems
    % The following theorem style uses a bold heading for the theorem and normal (upright) text. The environment begins with a triangle of color ›stroke1‹ pointing to the right and uses a QED symbol that is a triangle of the same color pointing to the left. Thus, the environment is enclosed by triangles.
    \declaretheoremstyle
    [
        spaceabove = \topsep,
        spacebelow = \topsep,
        headfont = \bfseries,
        headformat = \textcolor{stroke1}{$\blacktriangleright$} \NAME~\NUMBER \NOTE,
        notefont = \bfseries,
        notebraces = {(}{)},
        bodyfont = \normalfont,
        postheadspace = 0.5 em,
        qed = \textcolor{stroke1}{\bfseries$\blacktriangleleft$},
    ]
    {myTheoremStyle}
    
    % The QED symbol used in proofs is a squre with color ›stroke1‹ in order to look similar to the theorem environments.
    \renewcommand*{\qedsymbol}{\textcolor{stroke1}{$\blacksquare$}}
    
    \declaretheorem
    [
        style = myTheoremStyle,
        name = Conjecture,
        numberwithin = chapter,
    ]
    {conjecture}
    \declaretheorem
    [
        style = myTheoremStyle,
        name = Proposition,
        sharenumber = conjecture,
    ]
    {proposition}
    \declaretheorem
    [
        style = myTheoremStyle,
        name = Claim,
        sharenumber = conjecture,
    ]
    {claim}
    \declaretheorem
    [
        style = myTheoremStyle,
        name = Lemma,
        sharenumber = conjecture,
    ]
    {lemma}
    \declaretheorem
    [
        style = myTheoremStyle,
        name = Corollary,
        sharenumber = conjecture,
    ]
    {corollary}
    \declaretheorem
    [
        style = myTheoremStyle,
        name = Theorem,
        sharenumber = conjecture,
    ]
    {theorem}
    \declaretheorem
    [
        style = myTheoremStyle,
        name = Definition,
        sharenumber = conjecture,
    ]
    {definition}
    \declaretheorem
    [
        style = myTheoremStyle,
        name = Example,
        sharenumber = conjecture,
    ]
    {example}
    \declaretheorem
    [
        style = myTheoremStyle,
        name = Remark,
        sharenumber = conjecture,
    ]
    {remark}
\else
    % This is the default style. That is, the head is bold, the rest is italic, and there is no symbol to denote the end of the environment.
    \theoremstyle{plain}
    
    \newtheorem{conjecture}{Conjecture}[chapter]
    \newtheorem{proposition}[conjecture]{Proposition}
    \newtheorem{claim}[conjecture]{Claim}
    \newtheorem{lemma}[conjecture]{Lemma}
    \newtheorem{corollary}[conjecture]{Corollary}
    \newtheorem{theorem}[conjecture]{Theorem}
    \newtheorem{definition}[conjecture]{Definition}
    \newtheorem{example}[conjecture]{Example}
    \newtheorem{remark}[conjecture]{Remark}
\fi


%%%%%%%%%%%%%%%%%%%
%% Meta Commands %%
%%%%%%%%%%%%%%%%%%%

% A template for a function that can use an optional variable bracket size. Its arguments have the following meaning:
%   1. The name of the function.
%   2. The type of the left bracket. This should be a bracket symbol, as it will be forwarded to the command \left.
%   3. The type of the right bracket. The same restrictions as with parameter 2 hold here.
%   4. The arguments that the function takes, that is, the things that are enclosed by the brackets.
%   5. The size of the brackets. This should be a value like \big or similar, as it will be forwarded to the command \left.
\NewDocumentCommand{\functionTemplate}{m m m m o}%
{%
    \IfNoValueTF{#5}%
    {%
        \mathOrText{#1\left#2{#4}\right#3}%
    }%
    {%
        \mathOrText{#1#5#2{#4}#5#3}%
    }%
}

% The following two commands are used as variables for the following command.
\newcommand*{\leftBracketType}{(}
\newcommand*{\rightBracketType}{)}

% This is a command that creates a command that is a function as defined by the command \functionTemplate. Its arguments have the following meaning:
%   1. The name of the function command.
%   2. The name of the function itself.
%   3. The type of the left bracket. This will be forwarded to parameter 2 of \functionTemplate. The default is (. Use \lbrack for [ and \{ for }.
%   4. The type of the right bracket. This will be forwarded to parameter 3 of \functionTemplate. The default is ). The rest is similar to parameter 3.
% The command created has two optional arguments, which are as follows:
%   1. The arguments of the function. If this is empty, only the name of the function will be used.
%   2. The size of the brackets. This will be forwarded to parameter 5 of \functionTemplate.
\NewDocumentCommand{\createFunction}{m m o o}%
{%
    \renewcommand*{\leftBracketType}{\IfNoValueTF{#3}{(}{#3}}%
    \renewcommand*{\rightBracketType}{\IfNoValueTF{#4}{)}{#4}}%
    \NewDocumentCommand{#1}{o o}%
    {%
        \IfNoValueTF{##1}%
        {%
            \mathOrText{#2}%
        }%
        {%
            \functionTemplate{#2}{\leftBracketType}{\rightBracketType}{##1}[##2]%
        }%
    }%
}

% A template for a probabilistic symbol, which can make use of a condition denoted by |. Its arguments have the following meaning:
%   1. The name of the function.
%   2. The argument of the function.
%   3. The condition of the function. The default is that there is no condition.
%   4. The size of the brackets. This will be forwarded to parameter 5 of \functionTemplate.
\DeclareDocumentCommand{\probabilisticFunctionTemplate}{m m O{} o}
{%
    \functionTemplate{#1}%
    {\lbrack}%
    {\rbrack}%
    {#2\IfEmptyTF{#3}{}{\ \IfNoValueTF{#4}{\left}{#4}\vert\ \vphantom{#2}#3\IfNoValueTF{#4}{\right.}{}}}%
    [#4]%
}


%%%%%%%%%%%%%%%%%%%%%
%% Common Commands %%
%%%%%%%%%%%%%%%%%%%%%

%%%%%%%%%%%%%%%%%%%%%
% Number Sets

% Number sets appear in bold by default. The other option is to make them appear in blackboard bold.
\ifboldNumberSets
    \newcommand*{\N}{\mathOrText{\mathbf{N}}}
    \newcommand*{\Z}{\mathOrText{\mathbf{Z}}}
    \newcommand*{\Q}{\mathOrText{\mathbf{Q}}}
    \newcommand*{\R}{\mathOrText{\mathbf{R}}}
    \newcommand*{\C}{\mathOrText{\mathbf{C}}}
    \newcommand*{\indicatorFunctionSymbol}{\mathbf{1}}
\else
    \newcommand*{\N}{\mathOrText{\mathds{N}}}
    \newcommand*{\Z}{\mathOrText{\mathds{Z}}}
    \newcommand*{\Q}{\mathOrText{\mathds{Q}}}
    \newcommand*{\R}{\mathOrText{\mathds{R}}}
    \newcommand*{\C}{\mathOrText{\mathds{C}}}
    \newcommand*{\indicatorFunctionSymbol}{\mathds{1}}
\fi

%%%%%%%%%%%%%%%%%%%%%
% Probabilistic Functions
% All of these functions follow the outline of \probabilisticFunctionTemplate. That is, the syntax is, for example, \Pr{A}[B][\big], which would be shown as Pr[A | B] with \big brackets.

% Probability measure
\RenewDocumentCommand{\Pr}{m O{} o}%
{%
    \probabilisticFunctionTemplate{\mathrm{Pr}}{#1}[#2][#3]%
}

% Expected value
\NewDocumentCommand{\E}{m O{} o}%
{%
    \probabilisticFunctionTemplate{\mathrm{E}}{#1}[#2][#3]%
}

% Variance
\NewDocumentCommand{\Var}{m O{} o}%
{%
    \probabilisticFunctionTemplate{\mathrm{Var}}{#1}[#2][#3]%
}

%%%%%%%%%%%%%%%%%%%%%
% Landau Notation
% The following commands all take a mandatory argument, which is the term of the Landau notation, as well as an optional argument, which determines the size of the brackets.

% Big O
\DeclareDocumentCommand{\bigO}{m o}%
{%
    \functionTemplate{\mathrm{O}}{(}{)}{#1}[#2]%
}

% Small O
\DeclareDocumentCommand{\smallO}{m o}%
{%
    \functionTemplate{\mathrm{o}}{(}{)}{#1}[#2]%
}

% Big Theta
\DeclareDocumentCommand{\bigTheta}{m o}%
{%
    \functionTemplate{\upTheta}{(}{)}{#1}[#2]%
}

% Big Omega
\DeclareDocumentCommand{\bigOmega}{m o}%
{%
    \functionTemplate{\upOmega}{(}{)}{#1}[#2]%
}

% Small Omega
\DeclareDocumentCommand{\smallOmega}{m o}%
{%
    \functionTemplate{\upomega}{(}{)}{#1}[#2]%
}

%%%%%%%%%%%%%%%%%%%%%
% Constants

% Pi; ratio of a circle’s circumference to its diameter
\newcommand*{\circlePi}{\mathOrText{\uppi}}

% Euler’s constant. This command takes an optional parameter, which becomes the exponent of this constant.
\DeclareDocumentCommand{\eulerE}{o}%
{%
    \mathOrText{\mathrm{e}\IfNoValueTF{#1}{}{^{#1}}}%
}

% i; the imaginary unit
\newcommand*{\imaginaryUnit}{\mathOrText{\mathrm{i}}}

%%%%%%%%%%%%%%%%%%%%%
% Other

% A polynomial function. The mandatory parameter is the argument of the function, the optional one is the size of the brackets.
\DeclareDocumentCommand{\poly}{m o}%
{%
    \functionTemplate{\mathrm{poly}}{(}{)}{#1}[#2]%
}

% The identity function
\createFunction{\id}{\mathrm{id}}

% An indicator function. The first parameter is set as an index, the second is the argument of the function, and the third is the size of the brackets.
\NewDocumentCommand{\ind}{m o o}%
{%
    \IfNoValueTF{#2}%
    {%
        \mathOrText{\indicatorFunctionSymbol_{#1}}%
    }%
    {%
        \functionTemplate{\indicatorFunctionSymbol_{#1}}{(}{)}{#2}[#3]%
    }%
}

% The domain of a function. Its parameters are the same as for \poly.
\DeclareDocumentCommand{\dom}{m o}%
{%
    \functionTemplate{\mathrm{dom}}{(}{)}{#1}[#2]%
}

% The range of a function. Its parameters are the same as for \poly.
\DeclareDocumentCommand{\rng}{m o}%
{%
    \functionTemplate{\mathrm{rng}}{(}{)}{#1}[#2]%
}

% The d for an integral. The optional parameter becomes the exponent/degree of the operator.
\DeclareDocumentCommand{\d}{o}%
{%
    \mathrm{d}\IfNoValueTF{#1}{}{^{#1}}%
}

% A command that creates sets. The first parameter is the left-hand side, the second is the right-hand side, and the third (optional) parameter is the size of the brackets.
\DeclareDocumentCommand{\set}{m m o}%
{
    \mathOrText{\IfNoValueTF{#3}{\left}{#3}\{#1\ \IfNoValueTF{#3}{\left}{#3}\vert\
    \vphantom{#1}#2\IfNoValueTF{#3}{\right.}{}\IfNoValueTF{#3}{\right}{#3}\}}
}      % Contains newly defined commands useful for mathematics.
% This is where all the commands should go that you want to define yourself. % This is where user-defined commands should go.


% This is the thesis. The front matter as well as the references should not be changed. The main matter can be edited freely.
\begin{document}


    \frontmatter
    % This file contains the layout of the title page. It is a generous interpretation of the demo page provided by the MNF of the University of Potsdam.

% This page uses a different geometry, as the content will be centered (not including the binding correction).
\ifprintVersion
    \ifprofessionalPrint
        \newgeometry
        {
            textwidth = 134 mm,
            textheight = 220 mm,
            top = 38 mm + \extraborderlength,
            inner = 38 mm + \mybindingcorrection + \extraborderlength,
        }
    \else
        \newgeometry
        {
            textwidth = 134 mm,
            textheight = 220 mm,
            top = 38 mm,
            inner = 38 mm + \mybindingcorrection,
        }
    \fi
\else
    \newgeometry
    {
        textwidth = 134 mm,
        textheight = 220 mm,
        top = 38 mm,
        inner = 38 mm,
    }
\fi

% The format of the title page.
\begin{titlepage}
    \sffamily
    \begin{center}
        \vfil
        {\LARGE
            \rule[1 ex]{\textwidth}{1.5 pt}
            \onehalfspacing\printTitleBold\\[1 ex]
            {\vspace*{-1 ex}\Large \printGermanTitle}\\
            \rule[-1 ex]{\textwidth}{1.5 pt}
        }
        \vfil
        {\Large\textbf{\printAuthor}}
        \vfil
        {\large Exposé zur Universitäts\colloquialDegreeNameLowercase arbeit}\\[0.25 ex]
        \bigskip
        {\Large \colloquialDegreeName{} of Science}\\[0.5 ex]
        {\large\emph{(\degreeAbbreviation\,Sc.)}}\\
        \bigskip
        {\large im Studiengang\\[0.25 ex]
        \printProgram}
        \vfil
        {\large eingereicht am \printDateReceived{} am\\[0.25 ex]
        Fachgebiet Big Data Analytics der\\[0.25 ex]
        Fakultät Mathematik und Informatik\\[0.25 ex]
        der Philipps-Universität Marburg}
    \end{center}
    
    \vfil
\end{titlepage}

\restoregeometry

    \pagestyle{plain}

    \addchap{Abstract}
    % This file should contain the English abstract.



    \selectlanguage{ngerman}
    \addchap{Zusammenfassung}
    % This file should contain the German abstract.


    \selectlanguage{american}

    \addchap{Acknowledgments}
    % Here you can write whom you want to thank.



    \setuptoc{toc}{totoc}
    \tableofcontents

    \pagestyle{headings}
    \mainmatter

    \chapter{Introduction}
    % Absatz warum automatische Methoden wichtig sind.
% More Data makes it impossible for humans to manually find dependencies.
% Data Growth over years
The amount of data that is being generated is growing constantly and at an ever increasing pace. All digital data is estimated to double about every two years \cite{gantz2012digital}. This thesis will be focused around structured data, a subset of all digital data. Structured data refers to a type of data that is organized and formatted in a consistent manner, allowing for efficient search, retrieval, and analysis. More precisely, we will examine relational data, which is a type of structured data that is organized into tables or relations, where each table represents a set of entities or objects, and each row or tuple in the table represents a single instance of those entities. Example of efforts to collect relational data from the internet are the \textit{Web Table Corpora} \footnote{https://webdatacommons.org/webtables/} or \textit{Wikitables} \footnote{http://websail-fe.cs.northwestern.edu/TabEL/}. These project collect tables but do provide insights that could be extracted from the data. There is also publicly available data from governments, other researchers and private businesses.\\

\noindent One of the most fundamental concepts in relational data are foreign key relations\cite{casanova1982inclusion}. Foreign keys are a crucial aspect of relational databases as they help define the relationships between tables, maintain referential integrity, prevent errors, and improve the performance of operations pulling data from linked tables. They ensure that each record in one table corresponds to a valid record in another table, thereby promoting consistency and accuracy in the database. While foreign keys are not mandatory, they play a vital role in establishing clear relationships between tables and validating data as rows are added, updated, or removed. By linking data between tables, new insights can be extracted and previously hidden knowledge might get reviled. In today's economy, data profiling and therefore also the discovery of foreign key (and further inclusion dependencies), is a necessity which, if done by human experts, is connected to huge cost \cite{halevy2006data}.\\

\noindent In the ever-expanding field of data management and analytics, the accurate representation and comprehension of relationships within datasets stand as central challenge. Inclusion dependencies encapsulate hierarchical connections between attributes, playing a crucial role in the integrity and normalization of data \cite{casanova1982inclusion}. Understanding and discovering these dependencies have far-reaching implications for various applications, including database design \cite{levene2000justification}, query optimization \cite{gryz1998query}, and data quality assurance. As the volume and complexity of data continues to increase, there is an ever growing need for advanced methodologies and tools that can extract inclusion dependencies inherent in datasets. \\

\noindent This master thesis embarks on a comprehensive exploration of inclusion dependency discovery. Inclusion dependencies capture the relationships between attributes by specifying that the values in one set of attributes must be included in another. While traditional database design principles rely on the normalization process to ensure the minimization of redundancy and enhance data integrity, the discovery and exploitation of inclusion dependencies provide a nuanced perspective on data relationships, offering insights that extend beyond conventional normalization techniques. We will not only discuss state of the art algorithms but further expand the search to partial inclusion dependencies. This special kind of inclusion dependencies allows for (small) errors and opens the door for finding inclusions dependencies in non-perfect datasets with human errors, spelling differences or historically grown deviations. \\

\noindent Insights generated through different partial thresholds are not merely academic, they have practical implications for companies and governments alike. If a partial inclusion dependency at a $99\%$ threshold is found, organizations could use this information to check for impurity in the given attributes.


    %%%%%%%%%%%%%%%%%%%%%%%%%%%%%%%%%%%%%%%%%%%%%%%%%
    %% Please add the content of your thesis here. %%
    %%%%%%%%%%%%%%%%%%%%%%%%%%%%%%%%%%%%%%%%%%%%%%%%%

    \part{Foundation}

    \chapter{Foundations}
% TODO change definitions to follow some book

To ensure that the content of this thesis is readable by both experts and a general audience we need to formulate notations and definitions. These will reappear multiple times within the thesis and they are needed to formulate precise observations and draw conclusions. This sections sticks to the notation introduced by De Marchi et al. \cite{marchi2009unary}. \\

\noindent A relational instance $r$ of a relational schemata $R$ carries tuples of values, typically donated as $u$ or $v$. Using an attribute list taken from $R$, typically denoted as $X$ or $Y$, we can perform a projection on $R$, thereby selecting a subset of attributes. We notate it by $R[X]$. The same is possible for tuples $u$. Writing $u[X]$ references the selection of values in the tuple. \\

\begin{restatable}[Schemata and Attributes]{example}{schema}\label{exmp:schema}
% Rewrite
A relational instance can be thought of as a table where the schema is the structure of that table. An attribute can be imagined as a column in some table. If you think about picking multiple columns of some table, that would be an attribute list ($X$). Now every row has more than one value (a tuple of values, $u$), but every row has the exact same number of values (all tuple cardinalities are equal).
\end{restatable}

\noindent Inclusion Dependencies (INDs) represent a fundamental concept, denoting formal relationships between attributes in a database schema. An IND specifies that the values within one set of attributes are inherently included within the values of another set of attributes.

\begin{definition}[Inclusion Dependencies]\label{def:inds}
    Given two relational instances $r_i$ from $R_i$ and $r_j$ from $R_j$. An IND is defined as $R_i[X] \subseteq R_j[Y] \iff \forall \: u \in r_i[X], \exists \: v \in r_j[Y] \text{ such that } u[X] = v[Y]$. This condition can only hold if the cardinality of $X$ is equal to the cardinality of $Y$. We further call the left hand side (here $R_i[X]$) the dependent attribute(s) and the right hand side the referenced attribute(s).
\end{definition}

\begin{restatable}[Inclusion Dependencies]{example}{IND}\label{exmp:IND}
    The Tables \ref{tab:relExamp} show two small examples of relational data. The left table ($r_1$) has the attributes \textit{ID}, \textit{Name}, \textit{State} and \textit{Age}. The right table ($r_2$) has the attributes \textit{Name}, \textit{State} and \textit{Color}. We find the INDs $r_1[Name] \subseteq r_2[Name]$ and $r_2[State] \subseteq r_2[State]$. If $r_1$ had more entries and the \textit{ID} would just keep counting up, $r_1[Age] \subseteq r_1[ID]$ would be come valid eventually. There are no INDs where the attribute cardinality is greater than one.
\end{restatable}

\begin{table}[]
    \centering
    \begin{tabular}{c|c|c|c}
        ID & Name & State & Age \\
        \hline
        1 & Robin & HE & 22 \\
        2 & Hannah & BW & 24 \\
        3 & Christian & HE & 36 \\
        4 & Jakob & BW & 24 \\
        5 & Luka & BE & 23 \\
        6 & Mareike & BE & 22 \\
        7 & Leon & BE & 27 \\
    \end{tabular}
    \begin{tabular}{c|c|c}
        Name & State & Color \\
        \hline
        Jakob & BW & Green \\
        Mareike & BE & Purple \\
        Christian & HE & Red \\
        Leon & SH & Orange \\
    \end{tabular}
    \caption{Two example relational instances $r_1$ (left) and $r_2$ (right).}
    \label{tab:relExamp}
\end{table}

\noindent The complexity of discovering inclusion dependencies forms one of the hardest problems in computer science. More precisely, the discovery of all inclusion dependencies is W[3]-hard \cite{blasius2017parameterized}. This makes IND discovery one of the hardest problems in computer science. The number of possible candidates for each attribute size can be calculated. Notice that the formula below assumes that all IND of layers before where valid. In natural language, we search for the number of attribute combinations where each attribute is present at most once, allowing all permutations.

\begin{definition}[Candidate Space]\label{def:candidates}
    Let $\alpha$ be the fixed integer size of all possible $X_i$. Let $m$ be the number of attributes. Let $k$ be the number of possible candidates ($X_i \cap X_j = \emptyset$ where $i \not = j$ given $\alpha$).
    \[
        k = \underbrace{\binom{m}{\alpha}\cdot \alpha ! }_{\text{Left hand side combinations}} \cdot \underbrace{\binom{m-\alpha}{\alpha}\cdot \alpha!}_\text{Right hand side combinations}.
    \]
    This formula holds if $\alpha \leq \lfloor \frac{n}{2} \rfloor$ else $k$ is $0$.
\end{definition}

\noindent In a multi schema setting this calculation becomes more difficult. We now need to consider which schema can form which inter schema candidates while allowing intra schema candidates.
\begin{definition}[Candidate Space (Multi-Schema)]\label{def:candidates-MS}
    We will first define a function $q_r(\alpha, \tau)$, which computes the candidates a single relation ($r$) can produce, given $\alpha$, the size of combinations and $\tau$ an indicator, whether or not the schema was already used for the other side.

    \begin{align*}
        q_r(\alpha, \tau) = \begin{cases}
            0, & \text{if } \alpha > |r| \bigvee (\alpha > \lfloor\frac{|r|}{2}\rfloor \bigwedge \tau)\\
            \binom{|r|}{\alpha}\cdot \alpha!, & \text{if not } \tau \\
            \binom{|r| - \alpha}{\alpha}\cdot \alpha!, & \text{if } \tau
        \end{cases}
    \end{align*}
    Using this helper function we can calculate the possible candidates $k$ for a fixed $\alpha$ over schema $r_1, \dots, r_n$ by computing:
    $$
        k = \sum\limits_{i = 1}^n q_{r_i}(\alpha, FALSE) \cdot \sum\limits_{i = 1}^n q_{r_i}(\alpha, TRUE).
    $$
\end{definition}

\begin{definition}[Partial Inclusion Dependencies]\label{def:pinds}
    A partial inclusion dependency (pIND) is written as $r_1[X] \subseteq_{\rho} r_2[Y]$ where $\rho \in (0, 1]$ and the reaming notation is analog to Definition \ref{def:inds}. Here, the $\rho$ interval is not including $0$ since this would mean everything is a pIND of everything else, which is a trivial case. Further this notation refers to lists of tuples and takes the cardinality of duplicates into consideration. For the pIND $r_1[X] \subseteq_{\rho} r_2[Y]$ to be valid
    $$
        \frac{|r_1[X] \cap r_2[Y]|}
            {|r_1[X]|} \geq \rho
    $$
    needs to be true.
\end{definition}

\noindent 
This definition uses the $\cap$ operator, which is usually used in set theory and defined for sets. 
The definitions adaptation of the operator is $r_1[X] \cap r_2[Y] = u \in r_1[X] \: | \: \exists \; v \in r_2[Y] : u = v$.
In natural language the operator conducts the operation \textit{"remove all elements from the dependent attribute, that are not present at least once in the referenced attribute"}. Therefore the referenced attribute could also be a set in a sense that the duplication distribution only matters for the dependent attribute.\\

\noindent In the proposed algorithms there should the option of considering duplicate cardinalities. If not explicitly mentioned otherwise this thesis always refers to partial inclusions that consider duplicate cardinality.

\begin{restatable}[Partial Inclusion Dependency Properties]{theorem}{pInd}\label{theo:pInd}
    Like inclusion dependencies, partial inclusion dependencies also fulfill the reflexive rule. For any $\rho \in (0, 1]$ the partial inclusion dependency $r_i[X_j] \subseteq_{\rho} r_i[X_j]$ is valid.
    \begin{align*}
        \frac{|r_i[X_j] \cap r_i[X_j]|}
            {|r_i[X_j]|} & \geq \rho \\
        \frac{|r_i[X_j]|}
            {|r_i[X_j]|} & \geq \rho \\
            1 & \geq \rho
     \end{align*}
     Since $\rho$ is upper bounded by $1$ the last statement will always be true. \\

     \noindent Contrary to INDs, pINDs do not generally respect transitivity if $\rho < 1$. We will proof this claim by contradiction. Assume $r_1[X] = [1, 2, ..., 100], r_2[Y] = [2, ..., 1000], r_3[Z] = [10, 11, ..., 1000],$. If transitivity would hold for any $\rho$, then we should find that for $\rho \in (0, 1]$ where $r_1[X] \subseteq_\rho r_2[Y], r_2[Y] \subseteq_\rho  r_3[Z]$ are valid, $ r_1[X] \subseteq_\rho  r_3[Z]$ also needs to be valid. For the given example, if $\rho = 0.95$, we find a contradiction. \\

     \noindent Lastly, INDs and also pINDs respect projection. We will now outline a proof for this claim. Consider the attributes $r_1[X], r_2[Y], r_3[Z], r_4[W]$ where $r_1[X]$ and $r_2[Y]$ are in the same relation and $r_3[Z]$ and $r_4[W]$ are in the same relation. Assume $r_1[X], r_2[Y] \subseteq_\rho r_3[Z], r_4[W]$ is valid for some $\rho \in (0, 1]$. If projection holds, this implies that $r_1[X] \subseteq_\rho r_3[Z]$ and $r_2[Y] \subseteq_\rho r_4[W]$ have to be valid as well. If we now only consider the portion (with reduced size $\rho\%$) which satisfies $r_1[X], r_2[Y] \subseteq_1 r_3[Z], r_4[W]$ we can use the known properties for INDs and conclude that for at least $\rho\%$ $r_1[X] \subseteq_1 r_3[Z]$ and $r_2[Y] \subseteq_1 r_4[W]$ has to be valid. This also directly implies that $r_1[X] \subseteq_\rho r_3[Z]$ and $r_2[Y] \subseteq_\rho r_4[W]$ will be true if the remaining $(1-\rho)\%$ values are added again. This property is very important for search space pruning, which is the single most important task for (p)IND discovery \cite{liu2010discover}.
\end{restatable}

\begin{restatable}[Partial Inclusion Dependencies]{example}{pInd}\label{exmp:pInd}
    Let us consider the two attributes $r_1[X] = [1, 2, 3, 4]$ and $r_2[Y] = [1, 1, 2, 3]$. First we can conclude, that $r_2[Y] \subseteq r_1[X]$ is an inclusion dependency, since all values present in $r_2[Y]$ also occur in $r_1[X]$. This also directly causes $r_2[Y] \subseteq_{1.0} r_1[X]$ to be a valid pIND. We will now calculate the maximal partial thresholds. Like inclusion dependencies, partial inclusion dependencies are not symmetrical, this requires us to perform two calculations. We already discovered $r_2[Y] \subseteq_{1.0} r_1[X]$, which implies the maximal threshold is $1$. Let us use the proposed formula for the other direction.
    \begin{align*}
        \frac{|r_1[X] \cap r_2[Y]|}
            {|r_1[X]|} & \geq \rho \\
        \frac{|[1, 2, 3, 4] \cap [1, 2, 3]|}
            {|[1, 2, 3, 4]|} & \geq \rho \\ 
        \frac{|[1, 2, 3]|}
            {|[1, 2, 3, 4]|} & \geq \rho \\
        \frac{3}{4} & \geq \rho. \\ 
    \end{align*}
    We have found that when the threshold $\rho$ is less or equal to $0.75$, $r_1[X] \subseteq_{\rho} r_2[Y]$ is valid.
\end{restatable}

\noindent The complexity of discovering partial inclusion dependencies is inherited form the base problem. Since the complexity is calculated under a worst case assumption, it does not change when switching into the partial setting. The thesis will discuss in detail how the time complexity behaves under varying thresholds.


        
    \section{Technologies}
    % In this document I want to talk about storage for relational databases SQL, csv files.
% Explain that I will develop in Java


    \section{Related Work}
    % This document should discuss passed approaches focusing on pros and cons

\section{Related Work}\label{sec:rel_work}

Inclusion dependencies (INDs) are a highly influential concept in both database research and practice, with a wide range of contributions and applications. The introduction already provided some insight into their diverse application areas of INDs. In this section, we will focus on the key achievements related to the implication problem of INDs. We will go over different algorithms and discuss their unique features. \\

In 1981 INDs started as a general notation of referential integrity, which was already a well established concept back then \cite{date1981referential}. Casanova et al. presented a paper on the inference rules of INDs\cite{casanova1982inclusion}. Three axioms where introduced: \textit{reflexivity}, \textit{transitivity} and \textit{projection and permutation}. The application of these rules to partial INDs will later be discussed in detail (Section \ref{theo:pInd}). The paper further proofed that the discovery of INDs is PSPACE-complete if there is no limit on the size of inclusions. Publications typically fall into three groups of algorithms, foreign key discovery algorithms, unary IND discovery and n-nary IND discovery \cite{papenbrock2017data}. \\

In 1995 Bell and Brockhausen \cite{bell1995discovery} propose a graph-based approach to represent the relationships between attributes, allowing for a more efficient exploration of the search space. The algorithm is initiated with a directed graph, wherein all possible edges, which could not be pruned by statistical measures, are included. A directed edge in the graph represents an inclusion dependency, which is read as the edge from $A_i$ to $A_j$ ($A_i \rightarrow A_j$) represents the IND $A_i \subseteq A_j$. It then proceeds to remove those edges that failed the IND check. To determine the validity of an edge, the algorithm checks for transitivity, which enables it to answer whether a dependency could exist between two variables based on their relationships with a third variable, which was tested previously. If it is ascertained that a dependency is impossible, the algorithm skips the test and directly removes the given edge, thereby reducing the overall computational cost. The approach presented by Bell and Brockhausen for unary IND discovery has both reusable aspects and downsides. The algorithm's candidate generation technique, which uses data statistics such as data types and min-max values, can be reused in other discovery algorithms. This preprocessing step reduces the number of candidates that need to be validated and further reduce the storage overhead needed to store candidates. However, the validation technique used in the algorithm, which relies on SQL join-statements and requires accessing the data on disk, is infeasible for larger candidate sets. This limits the scalability of the approach and makes it less practical for large-scale data sets. Additionally, the need to store the data in a database and access it on disk for validation can add to the computational cost and time required for the discovery process.\\

The \textit{SPIDER} algorithm is a disk-backed, all-column sort-merge join with early termination used for the discovery of inclusion dependencies \cite{bauckmann2006efficiently}. It sorts the values of each attribute, removes duplicate values, and writes the results to disk in the first phase. In the second phase, it performs the actual inclusion dependency discovery by using a pointer for each file and validating all candidates at the same time. A major advantage is, that in this setting every value only needs to be read a single time from disk, which greatly reduces the I/O bottleneck. The Spider algorithm has been the subject of experimental evaluation and is considered one of the efficient techniques for unary IND discovery. Still, there are drawback if the data set is too big to be sorted in main memory or if the number of simultaneously open files allowed by the OS system is reached \cite{papenbrock2015divide}. \\

In 2009 Bauckmann et al. also proposed a partialized version of $SPIDER$ in a section of the same paper. The authors found that there where surprisingly many partial inclusions (under a 5\% threshold) in their test data sets. To find partial inclusion dependencies, they first count how many distinct violations are present and in a second step consider the amount of duplicates for not included values. This means their algorithm does not immediately stop once a single validation has been found but only after an added counter surpasses a given threshold. The paper is not particular clear on how the number of duplicates is stored/retrieved and additionally does not analyse the computational effect of these changes and with the original source code being lost, this a approach can only be verified using a best guess approach.\\

The \textit{SAWFISH} algorithm \cite{kaminsky2023discovering}, published in 2023, is designed for identifying similarity inclusion dependencies (sINDs) within datasets, introducing a novel perspective on inclusion dependencies (INDs). While traditional INDs assume error-free data, \textit{SAWFISH} incorporates a similarity measure to accommodate minor errors like typos. Given a similarity threshold $\omega$ a sIND is valid if and only if for all tuples in the left hand side there exists a tuple in the right hand side which has at least a similarity of $\omega$ under a set similarity measure. The authors used the edit distance as well as the normalized edit distance. Through preprocessing, metadata generation, and a sliding-window approach, \textit{SAWFISH} successfully identifies and validates sIND candidates using an inverted index, providing a valuable tool for database applications despite dirty data challenges. \\



    \part{Partial Inclusion Dependencies}
    
    \chapter{Metadata Extraction and Storage}

    \chapter{(Partial) Inclusion Graph}
    
    \part{Results}
    \chapter{Datasets}
To understand the performance of the proposed algorithms it is crucial to perform testing on a variety of data sets. For this purpose we will gather some real word data sets. Further we will create synthetic data sets that aim on edge cases to see if the performance is strongly dependent on structural assumptions.

\section{Real World Data Sets}
There are many sources for csv or tsv files online. I have decided to gather data from the US Government\footnote{\href{https://data.gov}{data.gov}}, the European Union\footnote{\href{https://data.europe.eu}{data.europe.eu}}, Kaggle\footnote{\href{https://kaggle.com}{Kaggle.com}}, Musicbrainz\footnote{\href{https://musicbrainz.org/}{musicbrainz.org}}, and Eurostat\footnote{\href{https://musicbrainz.org/}{ec.europa.eu}}. Further data set sources may be added. Related research papers sometimes discuss the origin of the used data, do not discuss the structure of the data they use \cite{papenbrock2017data,bauckmann2006efficiently, dursch2019inclusion, rostin2009machine}. In order to understand the resulting algorithm performance, we believe it is crucial to examine the data which is tested against. In this section we will discuss the data used and later try to understand why an algorithms performance may vary over different test sets.

\section{Synthetic Data Sets}
To evaluate the proposed algorithms under detailed aspects, we will generate synthetic data sets. The strategies and claims are based on \cite{jordon2022synthetic} synthetic data can be defined as \textit{data that has been generated using a purpose-built mathematical model or algorithm, with the aim of solving a (set of) data science task(s).} While we will not try to train a model with the synthetic data, it is still of great use for us, since we have absolute knowledge about the underlying structures. The decision is based on the fact that there is a lot of real word data available, since open data is a growing market which expected to grow even further \cite{EUopenData}. Synthetic data on the other hand enables us to evaluate the algorithm performances on edge cases, which we may not be able to find in the selection of real world data sets. \\

\noindent To test certain edge cases of the proposed algorithms, we will construct various edge case data sets. The \textit{SameSame} dataset consists of 32 attributes and 250.000 records. Each attribute carries the numbers 1 to 250.000 in the natural order. This means every attribute is a (partial) inclusion dependency of every other attribute. The same obviously also holds for combinations of columns. We will now calculate the expected number if (p)INDs in each layer. Since all candidates are perfect matches, the chosen threshold $\rho$ will not influence the number of pINDs. Table % TODO add ref
shows the number of candidates/pINDs for the \textit{SameSame} dateset.
% TODO calculate INDs
The edge case to test here is, how well the algorithm can understand equality relations and prune the candidate space. While this may seem like an unlikely edge case we will also investigate how often this happens in real world data sets. \\
% write about real world structures

\noindent Another source of synthetic data will be the TPC (Transaction Processing Performance Council) Benchmarks \footnote{\href{https://www.tpc.org/}{tpc.org}}. The TPC Benchmarks are a set of standardized and vendor-neutral performance benchmarks used to evaluate the processing and database capabilities of different systems. These benchmarks are designed to model various types of workloads. The TPC-E benchmark, for example, models a brokerage firm with customers who generate transactions related to trades, account inquiries, and market research, while the TPC-C benchmark is intended to model a medium complexity online transaction processing workload, patterned after an order-entry system with skewed access within individual data types/relations. Using scaling factors, a user can define the size of the synthetic database themselves. This enables us to examine the algorithm performances in a very controllable setting.


    %%%%%%%%%%%%%%%%%%%%%%%%%%%%%%%%%
    %% End of adding your content. %%
    %%%%%%%%%%%%%%%%%%%%%%%%%%%%%%%%%


    % Add the following chapters not to the current ›part‹ but one level above instead.
    \makeatletter
        \def\toclevel@chapter{-1}
        \def\toclevel@section{0}
    \makeatother

    \chapter{Conclusions \& Outlook}
    % This is where you conclude your thesis.



    % Following are the files and commands for the bibliography and the author’s publications.
    \pagestyle{plain}

    \renewcommand*{\bibfont}{\small}
    \printbibheading
    \addcontentsline{toc}{chapter}{Bibliography}
    \printbibliography[heading = none]

    \addchap{Declaration of Authorship}
    I hereby declare that this thesis is my own unaided work. All direct or indirect sources used are acknowledged as references.\\[6 ex]

\begin{flushleft}
    Heidelberg, \today
    \hspace*{2 em}
    \raisebox{-0.9\baselineskip}
    {
        \begin{tabular}{p{5 cm}}
            \hline
            \centering\footnotesize\printAuthor
        \end{tabular}
    }
\end{flushleft}


\end{document}