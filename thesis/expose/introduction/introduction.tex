In the ever-expanding field of data management and analytics, the accurate representation and comprehension of relationships within datasets stand as central challenge. Inclusion dependencies, a fundamental aspect of relational databases \cite{casanova1982inclusion}, encapsulate hierarchical connections between attributes, playing a crucial role in the integrity and normalization of data. Understanding and discovering these dependencies have far-reaching implications for various applications, including database design \cite{levene2000justification}, query optimization \cite{gryz1998query}, and data quality assurance. As the volume and complexity of data continue to increase, there is an ever growing need for advanced methodologies and tools that can extract inclusion dependencies inherent in datasets. \\

\noindent This master thesis embarks on a comprehensive exploration of inclusion dependency discovery. Inclusion dependencies capture the relationships between attributes by specifying that the values in one set of attributes must be included in another. While traditional database design principles rely on the normalization process to ensure the minimization of redundancy and enhance data integrity, the discovery and exploitation of inclusion dependencies provide a nuanced perspective on data relationships, offering insights that extend beyond conventional normalization techniques. We will not only discuss state of the art algorithms but further expand the search to partial inclusion dependencies. This special kind of inclusion dependencies allows for (small) errors and opens the door for finding inclusions dependencies in non-perfect datasets with human errors or spelling differences. 