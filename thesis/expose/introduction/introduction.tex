% Absatz warum automatische Methoden wichtig sind.
% More Data makes it impossible for humans to manually find dependencies.
% Data Growth over years
The amount of data that is being generated is growing constantly and at an ever increasing pace. All digital data is estimated to double about every two years \cite{gantz2012digital}. This thesis will be focused around structured data. Structured data refers to a type of data that is organized and formatted in a consistent manner, allowing for efficient search, retrieval, and analysis. More precisely, we will examine relational data. Relational data refers to a type of structured data that is organized into tables or relations, where each table represents a set of entities or objects, and each row or tuple in the table represents a single instance of those entities. Example of efforts to collect relational data from the internet are the \textit{Web Table Corpora} \footnote{https://webdatacommons.org/webtables/} or \textit{Wikitables} \footnote{http://websail-fe.cs.northwestern.edu/TabEL/}. These project collect tables but do provide insights that could be extracted from the data. \\

\noindent One of the most fundamental concepts in relational data are foreign key relations. Foreign keys are a crucial aspect of relational databases as they help define the relationships between tables, maintain referential integrity, prevent errors, and improve the performance of operations pulling data from linked tables. They ensure that each record in one table corresponds to a valid record in another table, thereby promoting consistency and accuracy in the database. While foreign keys are not mandatory, they play a vital role in establishing clear relationships between tables and validating data as rows are added, updated, or removed. By linking data between tables, new insights can be extracted and previously hidden knowledge might get reviled. \\

\noindent In the ever-expanding field of data management and analytics, the accurate representation and comprehension of relationships within datasets stand as central challenge. Inclusion dependencies, a fundamental aspect of relational databases \cite{casanova1982inclusion}, encapsulate hierarchical connections between attributes, playing a crucial role in the integrity and normalization of data. Understanding and discovering these dependencies have far-reaching implications for various applications, including database design \cite{levene2000justification}, query optimization \cite{gryz1998query}, and data quality assurance. As the volume and complexity of data continue to increase, there is an ever growing need for advanced methodologies and tools that can extract inclusion dependencies inherent in datasets. \\

\noindent This master thesis embarks on a comprehensive exploration of inclusion dependency discovery. Inclusion dependencies capture the relationships between attributes by specifying that the values in one set of attributes must be included in another. While traditional database design principles rely on the normalization process to ensure the minimization of redundancy and enhance data integrity, the discovery and exploitation of inclusion dependencies provide a nuanced perspective on data relationships, offering insights that extend beyond conventional normalization techniques. We will not only discuss state of the art algorithms but further expand the search to partial inclusion dependencies. This special kind of inclusion dependencies allows for (small) errors and opens the door for finding inclusions dependencies in non-perfect datasets with human errors or spelling differences. \\

\noindent Insights generated through different partial thresholds are not merely academic; they have practical implications for companies and governments alike. If a partial inclusion dependency at a $99\%$ threshold is found, organizations could use this information to check for impurity in the given attributes.

% Analysis of Algorithm performance and understanding underlying data better