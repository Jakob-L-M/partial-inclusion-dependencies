\section{Future Work}
We are aware that there is still potential left to improve pIND discovery. A central observation in the complex datasets with large n-ary pINDs is, that the generated candidates eventually become equal or almost equal to the valid pINDs of some layer. An idea would be to jump a few layers ahead if such a thing happens since it strongly hint towards the existence of pINDs in a (much) deeper layer. In \textbf{ACNH} we find that from the sevens to the twelves layer the generated candidates where always all valid.

The published code behind the algorithm can also be improved further. The implementation mostly relies on standard java data structures. There are libraries, such as FastUtil\footnote{\url{https://fastutil.di.unimi.it/}}, which offer high performance versions of these structures and may be able to improve the performance. Serialization is also performed in a human-readable style, which is not optimal. Encoding the attribute ids and occurrences (integers and longs) in one and four byte blocks could decrease the file size drastically which could help when I/O bottlenecks occur.