% Absatz warum automatische Methoden wichtig sind.
% More Data makes it impossible for humans to manually find dependencies.
% Data Growth over years
\section{Complexity of (p)ind discovery}
The amount of data that is being generated is growing constantly and at an ever increasing pace. All digital data is estimated to double about every two years \cite{gantz2012digital}. Throughout this research we will focus on structured data, a subset of all digital data, which refers to a type of data that is organized and formatted in a consistent manner, allowing for efficient search, retrieval, and analysis. More precisely, we will examine relational data, which is a type of structured data that is organized into relations. Each relation holds a set of attributes, which hold a collection of values. Example of efforts to collect relational data from the internet are the \textit{Web Table Corpora} \footnote{https://webdatacommons.org/webtables/} or \textit{Wikitables} \footnote{http://websail-fe.cs.northwestern.edu/TabEL/}. These project collect tables but do provide insights that could be extracted from the data. Relational data in general originates from sources spanning governments, business and research facilities. \\
One of the most fundamental concepts in relational data are foreign key relations\cite{casanova1982inclusion}. They state that the values contained in a set of attribute of some relation are a subset of some other set of values contained in a different collection of attribute from a potentially different relation. Foreign keys are a crucial aspect of relational databases as they help define the relationships between relations, maintain referential integrity, prevent errors, and improve the performance of operations. They ensure that each record in one table corresponds to a valid record in another table, thereby promoting consistency and accuracy in the database. While foreign keys are not mandatory, they play a vital role in establishing clear relationships between tables and validating data as rows are added, updated, or removed. By linking data between tables, new insights can be extracted and previously hidden knowledge might get reviled. In today's economy, data profiling and therefore also the discovery of foreign key (and further inclusion dependencies), is a necessity which, if done by human experts, is connected to huge cost \cite{halevy2006data}.\\

\begin{table}
\parbox{.45\linewidth}{
\centering
\begin{tabular}{ccc}
\hline
a&b&c\\
\hline
\end{tabular}
\caption{Foo}
}
\hfill
\parbox{.45\linewidth}{
\centering
\begin{tabular}{ccc}
\hline
d&e&f\\
\hline
\end{tabular}
\caption{Bar}
}
\end{table}
In the ever-expanding field of data management and analytics, the accurate representation and comprehension of relationships within datasets stand as central challenge. Inclusion dependencies encapsulate hierarchical connections between attributes, playing a crucial role in the integrity and normalization of data \cite{casanova1982inclusion}. Understanding and discovering these dependencies have far-reaching implications for various applications, including database design \cite{levene2000justification}, query optimization \cite{gryz1998query}, and data quality assurance. As the volume and complexity of data continues to increase, there is an ever growing need for advanced methodologies and tools that can extract inclusion dependencies inherent in datasets. \\

In Figure \ref{fig:pIND_example} we find a simplistic example of relational data. In our example the relations \textit{Person} and \textit{House} are supposed to represent data which was gathered by some community regarding the villagers and the existing houses. Notice how there seems to be a spelling mistake in \textit{Torsten} for the \textit{Aurora Haven} building. Classic INDs will find the relations $\textit{Person}[\textit{Post-nominal}] \subseteq \textit{House}[\textit{Owner Post-N}]$ while pINDs would be able to discover $\textit{Person}[\textit{Name}, \textit{Post-nominal}] \subseteq \textit{House}[\textit{Owner Name}, \textit{Owner Post-N}]$ at a threshold of $85\%$.

Insights generated through different partial thresholds are not merely academic, they have practical implications for companies and governments alike. If a partial inclusion dependency at a $99\%$ threshold is found, organizations could use this information to check for impurity in the given attributes.