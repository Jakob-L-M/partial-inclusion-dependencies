% Absatz warum automatische Methoden wichtig sind.
% More Data makes it impossible for humans to manually find dependencies.
% Data Growth over years
\section{Complexity of (p)ind discovery}
The amount of data that is being generated is growing constantly and at an ever increasing pace. All digital data is estimated to double approximately every two years \cite{gantz2012digital}. Throughout this research, we will focus on structured data, a subset of all digital data, which refers to a type of data that is organized and formatted in a consistent manner, allowing efficient search, retrieval, and analysis \cite{gryz1998query}. More precisely, we will examine relational data, which is a type of structured data organized into relations. Each relation holds a set of attributes that hold a collection of values. Examples of efforts to collect relational data from the internet are the \textit{Web Table Corpora}\footnote{\href{https://webdatacommons.org/webtables/}{webdatacommons.org/webtables/} (Last Access: 30/06/2024)} or \textit{Wikitables}\footnote{\href{http://websail-fe.cs.northwestern.edu/TabEL/}{websail-fe.cs.northwestern.edu/TabEL/}  (Last Access: 30/06/2024)}. These initiatives gather relations, but do not offer the insights that can be derived from the data. Relational data typically comes from a variety of sources, including government agencies, businesses, and research institutions.

Among the most fundamental constructs in relational data profiling are inclusion dependencies (INDs), a super set of foreign key relationships \cite{casanova1982inclusion}. They assert that the values within a set of attributes from one relation form a subset of the values within another set of attributes from a possibly different relation. Foreign keys are a crucial aspect of relational databases as they help define relationships between relations, maintain referential integrity, prevent errors, and improve the performance of operations. They guarantee that every entry in one table matches a valid entry in another table, thus enhancing the consistency and precision of the database. Although foreign keys are not obligatory, they are instrumental in defining explicit relationships between tables and ensuring data validation during row insertion, updating, or deletion. By linking data between tables, new insights can be extracted and previously hidden knowledge might get revealed. In today's economy, data profiling and therefore also the discovery of foreign keys (and further INDs), are a necessity which, if done by human experts, is connected to huge cost \cite{halevy2006data}.\\

Within the rapidly evolving domain of data management and analytics, the precise representation and comprehension of interrelationships within datasets constitute a fundamental challenge. INDs encapsulate hierarchical links between attributes, thus playing a pivotal role in ensuring data integrity and normalization \cite{casanova1982inclusion}. The identification of these dependencies has significant implications for a multitude of applications, including database design \cite{levene2000justification}, query optimization \cite{gryz1998query}, and data quality assurance \cite{fan2008dependencies}. As the volume and complexity of data continues to grow, there is a simultaneously growing demand for sophisticated methodologies and tools capable of extracting the inherent inclusion dependencies within datasets.

\begin{table}[!t]
\parbox{.42\linewidth}{
\resizebox{0.42\columnwidth}{!}{%
\centering
\begin{tabular}{llll}
\hline
\footnotesize \textbf{Name}& \footnotesize \textbf{Id}& \footnotesize \textbf{Born}&\footnotesize \textbf{Died}\\
\hline
Sophie & 4 & 16 & 43 \\
Hannah & 1 & 8 & 92 \\
Angela & 2 & 6 & 30 \\
Angela & 3 & 7 & 29 \\
Sophie & 3 & 11 & 40 \\
Sophie & 2 & 6 & 10 \\
Angela & 1 & 0 & 12 \\
Sophie & 1 & 2 & 52 \\
\end{tabular}
}%
\caption{Person}
\label{tab:person}
}
\hfill
\parbox{.52\linewidth}{
\resizebox{0.52\columnwidth}{!}{%
\centering
\begin{tabular}{lll}
\hline
\footnotesize \textbf{Name}& \footnotesize \textbf{P\_Name}& \footnotesize \textbf{P\_Id}\\
\hline
Vale Villa & Angela & 2 \\
Serpent Spire & Hanna & 1 \\
Thunder Tower & Sophie & 3 \\
Rural Retreat & Angela & 1 \\
Aurora Haven & Sophie & 2 \\
Mirage Mansion & Angela & 3 \\
\end{tabular}
}%
\caption{House}
\label{tab:house}
}
\end{table}

In Tables \ref{tab:person} and \ref{tab:house} we find a simplistic example of relational data. \textit{Person} and \textit{House} are supposed to represent data collected by some community regarding the villagers and houses of that community. \textit{Person} carries the villagers names, their id and the years they where born and died in. The id column is used to distinguish between villagers with the same name. \textit{House} holds information on the buildings name and its owner which is stored in \textit{P\_Name} (person name) and \textit{P\_Id} (persons id). Notice how there seems to be a spelling mistake in \textit{Hanna} for the \textit{Serpent Spire} building. Classic INDs will find the relations $$\textit{House}[\textit{PPost}]\subseteq \textit{Person}[\textit{Post}] \textit{House}[\textit{PPost}]$$ while pINDs would be able to discover $$\textit{Person}[\textit{Name}, \textit{Post}] \subseteq_{0.85} \textit{House}[\textit{PName}, \textit{PPost}]$$ at a threshold of $85\%$. This simplistic example shows that insights generated through different partial thresholds are not merely academic, they have practical implications for companies and governments alike. If a partial inclusion dependency is found at a threshold of $99\%$, organizations could use this information to check for impurities in the given attributes. In the following, we will first discuss the foundations (Section~\ref{sec:foundations}) of pIND and define which properties of IND discovery hold for the discovery of pIND. We go over existing state-of-the-art algorithms and expand them to enable pIND discovery (Section \ref{sec:algo_partial}). These algorithms will be used as a reference point for later evaluation. Section \ref{sec:spind} will introduce our own proposal \textit{SPIND}. It solves major disadvantages of \textit{SPIDER} \cite{bauckmann2006efficiently} and \textit{BINDER} \cite{papenbrock2015divide} while offering stronger performances and more flexibility in both unary and nary settings. We then analyze our experimental results and examine pINDs within real-world datasets (Section \ref{sec:eval}). Finally, related research is discussed in Section \ref{sec:rel_work}.