% TODOs:
% properly include citations
\subsection{Null handling} \label{sec:null_handling}


Relational databases can include \textit{NULL} or \textit{EMPTY} values.
The presence of such values can come from various sources, such as historical database expansions, missing or lost data, unknown values, or entries that cannot be populated with any specific value. There are two primary interpretations for handling these values \cite{vassiliou1979null}.
Entries that are undefined and not relevant, and entries that are missing even though they are relevant. Using these interpretations, we will now define and discuss different approaches on how \textit{NULL} entries are treated. No approach is less valid than the other, but one needs to choose a treatment that fits the dataset. Our algorithm, \textit{SPIND}, provides greater flexibility than previous algorithms by allowing users to select any of the available \textit{NULL} handling techniques, unless otherwise specified.

\subsubsection{The \textbf{unknown} interpretation}
In this interpretation, the truth value of $x == y$ where $x$ or $y$ or both are \textit{NULL} is considered unknown \cite{codd1979extending}. It is not true or false, but lies under the assumption that the \textit{NULL} entry could be filled using some, possibly distinct value. This interpretation of \textit{NULL} entries causes inclusion or equality operations to also return neither true nor false, but also an \textit{unknown} truth statement. Our method for identifying pINDs requires a static dataset and must be executed again if there are any new or modified entries. Consequently, the \textit{unknown} interpretation does not provide useful information, rendering this interpretation impractical.

\subsubsection{The \textbf{subset} interpretation}\label{sec:null_subset} This interpretation is similar to an interpretation known as \textit{more informative tuples} \cite{zaniolo1982database}. A \textit{more informative tuple} relation is a relation between two tuples $t_1$ and $t_2$ with attributes $A$ and $B$ where for every $a \in A$ where $t_1[a] \not = NULL$, $a \in B$ and $t_1[a] = t_2[a]$ are valid. Under this notation, $t_2$ carries all the information of $t_1$, but potentially contains more information for the $NULL$ entries of $t_1$. Therefore, $t_2$ is (equally or) more informative than $t_1$. Given two relational instances $r_1$ and $r_2$ with attribute sets $X \in r_1$ and $Y \in r_2$ where $|X| = |Y|$. For a \textit{duplicateAware} setting, we define $r_1[X] \subseteq_\rho r_2[Y]$ as valid if there are at least $\lceil |r_1[X]| \cdot \rho \rceil$ tuples in $r_1[X]$ for which a \textit{more informative tuple} exists in $r_2[Y]$. Analogously, we require $\lceil |\{r_1[X]\}| \cdot \rho \rceil$ more informative tuples for a \textit{duplicateUnaware} setting. Practically, we ignore \textit{NULL} values in this setting and proceed as if they do not exist.

\subsubsection{The \textbf{foreign} interpretation}
This interpretation is a minor modification of the \textit{subset} interpretation. The key distinction is the additional restriction that the referenced side must not contain \textit{NULL} values. This interpretation is motivated by the need to identify foreign keys.

\subsubsection{The \textbf{equality} interpretation}
This interpretation is an optimistic version of the \textit{unknown} interpretation, treating all $NULL$ entries as equal. The result is equal to the subset approach if we consider $NULL$ as just another explicit value. Logically, this interpretation should be chosen if $NULL$ implicitly signifies a value which cannot be expressed explicitly (e.g. infinity in a numeric column), but still holds meaning and is expected in the referenced for a valid pIND.

\subsubsection{The \textbf{inequality} interpretation}
The pessimistic version of the \textit{unknown} interpretation considers every $NULL$ entry to be distinct from all others. Therefore, the comparison between two entries $x$ and $y$ where one or both are $NULL$ always yields false. We are in a setting where $NULL$ entries are considered not yet filled and we choose a worst-case scenario where the replacement values are all distinct from each other and all values of the relation itself. In a non-partial setting, an attribute containing at least one $NULL$ value cannot
be a dependent side for any IND. This interpretation is equal to the concept of a certain world \cite{kohler2013possible}. No matter how the $NULL$ values are set, the found pINDs will remain valid.

\subsubsection{\textbf{Possible World}}
In a possible world \cite{kohler2013possible}, all pINDs that could become valid under some replacement of $NULLs$ are considered valid. This creates a whole new complexity layer when trying to choose the optimal $NULL$ replacements. If a referenced attribute has $n$ $NULL$ entries, we could avoid $n$ violations. The optimal way to replace $NULL$ would be to choose the $n$ values with the highest occurrences over all dependent attributes that do not appear in the referenced attribute (occurrences counted as one for each attribute if we are in the \textit{duplicateUnaware} setting). That way we avoid the highest possible number of violations. For this to work, we would need to keep track of the highest $n$ occurring values in all dependent attributes. Since $n$ might be very large, we might not be able to store all values in main memory, and would require to store these values on disk. The described problems are certainly solvable, but are out of scope for our current research. For this reason, the algorithms do not offer pIND discovery in a possible world setting.