% TODOs:
% properly include citations

Relational databases can include \textit{NULL} or \textit{EMPTY} values.
The allowance of such values originates from % TODO: write why null makes sense
% The ANSI/SPARC interim report [1] cites 14 possible manifestations of null - ANSI/X3/SPARC Study Group on Data Base Management Systems, Interim Report, ANSI, February 1975. 
There are different approaches on how \textit{NULL} entries are treated. None of the approaches
is less valid than another, but one needs to choose a treatment. We will now discuss the
three common approaches on treating \textit{NULL} entries and discuss how they affect pIND discovery.

\subsection{The \textit{unknown} interpretation}
@article{codd1979extending,
  title={Extending the database relational model to capture more meaning},
  author={Codd, Edgar F},
  journal={ACM Transactions on Database Systems (TODS)},
  volume={4},
  number={4},
  pages={397--434},
  year={1979},
  publisher={ACM New York, NY, USA}
}
In the interpretation the truth value of $x = y$ where $x$ or $y$ or both are \textit{NULL} is considered as unknown.
It is not true or false, but rather lies under the assumption that the \textit{NULL} entry could be filled using some,
possibly distinct value. This interpretation of \textit{NULL} entries causes inclusion or equality operations to also
return neither true or false, but also an \textit{unknown} truth statement.
Our search for pINDs expects a static data input and would need to be re-run in the event of new or updated entries.
Therefor the \textit{unknown} interpretation does not provide us with usable inforation. We would rather have the
expert user decide between one of the following interpretations.

\subsection{The \textit{subset} interpretation}
This interpretation is similar to an interpretation known as \textit{more informative tuples}.
@inproceedings{zaniolo1982database,
  title={Database relations with null values},
  author={Zaniolo, Carlo},
  booktitle={Proceedings of the 1st ACM SIGACT-SIGMOD symposium on Principles of database systems},
  pages={27--33},
  year={1982}
}
A \textit{more informative tuple} relation is a relation between two tuples $t_1$ and $t_2$ with attributes $A$ and $B$
where for every $a \in A$ where $t_1[a] \not = NULL$, $a \in B$ and $t_1[a] = t_2[a]$ are valid. Under this notation,
$t_2$ carries the entire inforation of $t_1$, but potentially contains more inforation for the $NULL$ entries of $t_1$.

For pIND discovery we now consider a \textit{subset} interpretation. Given two relational instances $r_1$ and $r_2$ with attribute sets
$X \in r_1$ and $Y \in r_2$ where $|X| = |Y|$. We define $r_1[X] \subseteq_\rho r_2[Y]$ as valid if there are at least $\lfloor |r_1[X]| \cdot \rho \rfloor$
tuples in $r_1[X]$ for which a \textit{more informative tuple} exists in $r_2[Y]$.

\subsection{The \textit{foreign} interpretation}
A slight variation of the \textit{subset} interpretation. The difference being the added constraint that the referenced side is not allowed to
carry \textit{NULL} values. The motivation for this interpretation is the detection of foreign keys.
% TODO: Write how this only is a difference in the unary discovery, since the n-ary generation will automatically catch this constraint

\subsection{The \textit{equality} interpretation}
This interpretation could be considered a somewhat optimistic version of the \textit{unknown} interpretation in a sense that we consider all $NULL$ entries as being equal.
The result is equal to the other approaches, if we consider $NULL$ as just another explicit value.
Logically this interpretation should be chosen if $NULL$ implicitly defines an value which can not be expressed explicitly (e.g. infinity in a numeric column).

\subsection{The \textit{inequality} interpretation}
The pessimistic version of the \textit{unknown} interpretation. Here we consider every $NULL$ entry as an distinct value.
Therefor the comparison between two entries $x$ and $y$ where one or both are $NULL$ always yields false.
We are in a setting where $NULL$ entires are considered not yet filled and we choose a worst case scenario where the
replacement values are all distinct form each other and the relation itself.
In a non-partial setting this would restrict an attribute with at least one $NULL$ entry from forming and INDs. Applying
a partial setting, there is still a chance, that attributes with only a few $NULL$ entries are able to form pINDs.