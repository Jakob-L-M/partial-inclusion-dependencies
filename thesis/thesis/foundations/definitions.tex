\chapter{Foundations of pINDs}\label{sec:foundations}
This chapterexpands the notation introduced by De Marchi et al. \cite{marchi2009unary}. A relational instance $r$ of a relational schemata $R$ carries tuples of values, typically denoted as $u$ or $v$. Using an attribute list taken from $R$, typically denoted as $X$ or $Y$, we can perform a projection $r[X]$ on $r$, thereby selecting a subset of attributes. Similarly for tuples $u$, we write $u[X]$ referencing the selection of values in the tuple.

INDs represent a fundamental concept that denotes formal relationships between attributes in a database schema. An IND specifies that the values within one set of attributes are inherently included within the values of another set of attributes.
\begin{definition}[Inclusion dependencies]\label{def:inds}
    Given two relational instances $r_1$ from $R_1$ and $r_2$ from $R_2$. An IND is defined as $$r_1[X] \subseteq r_2[Y] \iff \forall \: u \in r_1[X], \exists \: v \in r_2[Y] : u[X] = v[Y].$$ This condition can only hold if the cardinality of $X$ is equal to the cardinality of $Y$. We further call the left hand side (here $r_1[X]$) the dependent attribute(s) and the right hand side the referenced attribute(s).
\end{definition}
The complexity of discovering INDs is one of the hardest problems in computer science. More precisely, the discovery of all INDs is \textit{W}[3]-hard, making it one of the hardest naturally occurring problems in computer science \cite{blasius2017parameterized}. 

% We can calculate potential IND candidates for each attribute cardinality, we will call the different cardinalities \textit{layers}. We will now calculate the worst case candidate space cardinality, for which we assume that all (p)IND candidates of the previous layers are valid. Let $X_i$ be all possible attribute lists for $r \in R$ of size $\alpha$, $m$ the number of total attributes, and $k$ the number of non-trivial candidates. For $\alpha \leq \lfloor \frac{n}{2} \rfloor$ we can evaluate
%\[
%    k = \underbrace{\binom{m}{\alpha}}_{\text{Left hand side combinations}} \cdot \underbrace{\binom{m-\alpha}{\alpha}\cdot \alpha!}_\text{Right hand side combinations},
%\]
%otherwise $k = 0$. We do not permute the left-hand side, since the resulting INDs would only yield redundant information.

%In a multi-schema setting, this calculation becomes more difficult. We now need to consider which schema can form which inter-schema candidates while allowing intra-schema candidates. We define 
%\begin{align*}
%    q_r(\alpha, \tau) = \begin{cases}
        % 0 & \text{if } \alpha > |r| \bigvee (\alpha > \lfloor\frac{|r|}{2}\rfloor \bigwedge \tau),\\
        % \binom{|r|}{\alpha}\cdot \alpha! & \text{if not } \tau, \\
        % \binom{|r| - \alpha}{\alpha}\cdot \alpha! & \text{if } \tau,
%    \end{cases}
%\end{align*}
%which computes the candidates a single relational instance ($r$) can produce, given $\alpha$, the size of combinations and $\tau$ an indicator, whether the schema was already used for the other side.

%Using this helper function we can calculate the number of possible candidates $k$ for a fixed $\alpha$ over relational instances $r_1, \dots, r_n$ from $R_1, \dots, R_n$ by computing:
%$$
%    k = \sum\limits_{i = 1}^n q_{r_i}(\alpha, \textit{false}) \cdot \sum\limits_{j = 1}^n q_{r_j}(\alpha, i == j).
%$$


\begin{definition}[Partial inclusion dependencies]\label{def:pinds}
    A pIND is written as $r_1[X] \subseteq_{\rho} r_2[Y]$ where $\rho \in (0, 1]$ and the remaining notation is analogous to Definition \ref{def:inds}. Our notation refers to lists of tuples and takes the cardinality of duplicates into consideration. The pIND $r_1[X] \subseteq_{\rho} r_2[Y]$ is valid if
    $$
        \frac{|[u \in r_1[X] \: | \: \exists \; v \in r_2[Y] : u = v]|}
            {|r_1[X]|} \geq \rho.
    $$
    If we do not consider duplicates, the condition simplifies
    $$
         \frac{|\{r_1[X]\} \cap \{r_2[Y]\}|}
            {|\{r_1[X]\}|} \geq \rho.
    $$
     The possible values for $\rho$ do not include $0$, as this would mean that everything is a pIND of everything else, which is a trivial case.
\end{definition}

As described in the definition of pINDs, a decision must be made regarding the treatment of duplicate values. Bauchmann et al. have already suggested two different versions \cite{bauckmann2006efficiently} which we will refer to as \textit{duplicateAware} and \textit{duplicateUnaware}.
For the \textit{duplicateAware} version, the maximum number of violations that can occur for $r_1[X] \subseteq_{\rho} r_2[Y]$ to be valid is equal to $\lfloor (1-\rho) \cdot |r_1[X]| \rfloor$. A violation refers to a value found on the left-hand side but not on the right-hand side. We examine whether less than $\rho$ percent of the total (including duplicates and \textit{NULLs}) entries would need to be changed. When we discover a violation during the validation process, the total violations increase by the occurrences of the violation causing value. Once the violations surpass the maximum number of violations, we disregard the given candidate. In a \textit{duplicateUnaware} setting we care about the changes in unique values, while not regarding their number of occurrences. We ask the question: Do less than $\rho$ percent of the unique values need to be modified, such that the IND $r_1[X] \subseteq r_2[Y]$ becomes valid? In this setting, the maximum number of violations is equal to $\lfloor(1-\rho) \cdot |\{r_1[X]\}| \rfloor$. Observe the use of the set notation around the attribute, which indicates that we are counting the distinct values. \\

\section{Properties of partial inclusion dependencies}\label{subsec:pIND_props}
Typically strategies for IND discovery rely on properties found in Armstrong's axioms \cite{armstrong2002armstrong}. We now examine which properties remain valid in a partial context. 

\subsubsection{\textbf{Reflexivity:}}
For any $\rho \in (0, 1]$ the pIND $r_i[X_j] \subseteq_{\rho} r_i[X_j]$ is valid. Simplifying the left-hand side of Definition \ref{def:pinds} yields to condition
$$
    \rho \leq \frac{|r_i[X_j] \cap r_i[X_j]|}
        {|r_i[X_j]|} = \frac{|r_i[X_j]|}
        {|r_i[X_j]|} = 1.
$$
Since $\rho$ is upper bounded by $1$ the pIND will always be valid. Therefore pINDs also adhere to reflexivity.

\subsubsection{\textbf{Transitivity:}}
Unlike INDs, pINDs generally do not adhere to transitivity. We will demonstrate this assertion through a proof-by-contradiction. Assume $r_1[X] = [1, 2, ..., 100]$, $r_2[Y] = [2, ..., 1000]$, and $r_3[Z] = [10, 11, ..., 1000]$. If transitivity would hold for any $\rho$, then we should find that for $\rho \in (0, 1]$ where $r_1[X] \subseteq_\rho r_2[Y]$ and $ r_2[Y] \subseteq_\rho  r_3[Z]$ are valid, $ r_1[X] \subseteq_\rho  r_3[Z]$ also needs to be valid. For the given example, if $\rho = 0.95$, we find a contradiction.

\subsubsection{\textbf{Projection:}}
Finally, both INDs and pINDs adhere projection. Consider the lists of tuples $r_1[X]$, $r_1[Y]$, $r_2[Z]$, $r_2[W]$, with $X \cap Y = \emptyset$ and $Z \cap W = \emptyset$. Assume $r_1[XY] \subseteq_\rho r_2[ZW]$ is valid for some $\rho \in (0, 1]$. If projection holds, this implies that $r_1[X] \subseteq_\rho r_2[Z]$ and $r_1[Y] \subseteq_\rho r_2[W]$ must also be valid. If we now only consider the portion which satisfies $\tilde r_1[XY] \subseteq_1 \tilde r_2[ZW]$ (with at least $|r_1| \cdot \rho$ entries) we can use the known properties for INDs and conclude that for $\tilde r_1[X] \subseteq_1 \tilde r_2[Z]$ and $\tilde r_1[Y] \subseteq_1 \tilde r_2[W]$ have to be valid. This also directly implies that $r_1[X] \subseteq_\rho r_2[Z]$ and $r_1[Y] \subseteq_\rho r_2 [W]$ will be true if the remaining $|r_1| \cdot (1-\rho)$ entries are added again. This characteristic is crucial for reducing the search space, which is the most critical task in the discovery of (p)INDs \cite{liu2010discover}. \\

\noindent The complexity of pIND discovery is inherited form IND discovery. Given that the complexity is assessed under the worst-case scenario, it remains unchanged when transitioning to the partial context.

