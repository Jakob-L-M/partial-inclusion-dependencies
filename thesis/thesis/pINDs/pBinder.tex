\section{Partial BINDER}
\textit{BINDER} is a IND discovery algorithm that uses a divide-and-conquer approach to efficiently find unary and nary INDs \cite{papenbrock2015divide}. It was shown that \textit{BINDER} outperforms other exact, non-distributed state-of-the-art algorithms in both unary and nary settings \cite{dursch2019inclusion}. Due to its strong performance, we also offer an adapted version of \textit{BINDER} which can handle partial IND discovery called \textit{pBINDER}. Our findings demonstrate that the discovery of pINDs does not significantly increase computational time. Furthermore, due to an optimized candidate generation, \textit{pBINDER} shows enhanced performance relative to \textit{BINDER} for nary pIND discovery.

\subsubsection{\textbf{Existing Code.}}
The authors supplied the original \textit{BINDER} source code, which will serve as an additional reference alongside \textit{SPIDER}. The partial version, \textit{pBINDER}, closely resembles \textit{BINDER} except for updates to libraries, refactoring, improved candidate generation (see Section \ref{sec:candidate_gen}), and subsequent validation modifications. 

\subsubsection{\textbf{Validation Adjustments.}}
The Validator of \textit{BINDER} handles the pruning of IND candidates. Initially, the algorithm assumes that all candidates are valid. Whenever we find a conflicting value, which is a value that is present on the dependent side but not on the referenced side of a candidate, the validator removes the given candidate. To consider partial INDs we need to expand \textit{BINDER} such that the validator keeps track of the number of violations and only removes a candidate if more violations than a given threshold occurred. The candidate violations are initialized analogously to the initialization of \textit{SPIDER}.

Note that \textit{pBINDER} only supports a \textit{duplicateAware} setting. \linebreak \textit{BINDER's} efficiency lies in a divide and conquer approach, where attributes are split into buckets using hash functions. For example, with $n$ buckets, we use \textit{hashCode} $\% \: n$ as the separation function. During validation, buckets are iterated in some order, and if an attribute is no longer in any candidate, we skip loading its bucket to save resources. However, \textit{BINDER} cannot count distinct values if not all buckets are loaded, especially before validation starts. Implementing \textit{duplicateUnaware} pIND discovery into \textit{pBINDER} would divert the algorithm too much from the original algorithm to a point where discussing the newly introduced complexity of pIND discovery would be unreasonable.